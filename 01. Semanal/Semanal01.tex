\documentclass{article}

% Symbols
\usepackage{amsfonts, amsthm}
\usepackage{upgreek}
\usepackage{physics}
\usepackage{cancel}
\usepackage{amssymb, latexsym, amsmath}

%Algorithms
\usepackage[ruled,lined,linesnumbered,commentsnumbered]{algorithm2e}

%% Identación
\setlength{\parindent}{0cm}

% Código
\newcommand{\code}[1]{\textcolor{white!25!black}{\texttt{#1}}}
\usepackage{listings}

%AMS
\usepackage{amsthm}
\newtheorem{algo-thm}{Algoritmo}

% Proof
\renewcommand*{\proofname}{\textbf{Demostraci\'on:}}
% Theorem
\newtheorem*{theorem}{Teorema}

% Graphics
\usepackage{graphicx}
\usepackage{pgf}

% Color a letras.
%\usepackage[usenames,dvipsnames,svgnames,table]{xcolor}

% Tikz
\usepackage{tkz-graph}
\usepackage{tikz}
\usetikzlibrary{arrows,automata}
\usepackage{tikz}
\usetikzlibrary{arrows,automata}
%\usetikzlibrary[topaths]

% Def. Dr. César.
\usetikzlibrary{shapes,calc}
\tikzstyle{edge}=[shorten <=2pt, shorten >=2pt, >=stealth, line width=1.1pt]
\tikzstyle{blueE}=[shorten <=2pt, shorten >=2pt, >=stealth, line width=1.5pt, blue]
\tikzstyle{blackV}=[circle, fill=black, minimum size=6pt, inner sep=0pt, outer sep=0pt]
\tikzstyle{blueV}=[circle, fill=blue, draw, minimum size=6pt, line width=0.75pt, inner sep=0pt, outer sep=0pt]
\tikzstyle{redV}=[circle, fill=red, draw, minimum size=6pt, line width=0.75pt, inner sep=0pt, outer sep=0pt]
\tikzstyle{redSV}=[semicircle, fill=red, minimum size=3pt, inner sep=0pt, outer sep=0pt, rotate=225]
\tikzstyle{blueSV}=[semicircle, fill=blue, minimum size=3pt, inner sep=0pt, outer sep=0pt, rotate=225]
\tikzstyle{blackSV}=[semicircle, fill=black, minimum size=3pt, inner sep=0pt, outer sep=0pt, rotate=225]
\tikzstyle{vertex}=[circle, draw, minimum size=6pt, line width=0.75pt, inner sep=0pt, outer sep=0pt]

% Margins
\addtolength{\voffset}{-1.5cm}
\addtolength{\hoffset}{-1.5cm}
\addtolength{\textwidth}{3cm}
\addtolength{\textheight}{3cm}

%Header-Footer
\usepackage{fancyhdr}
\renewcommand{\headrulewidth}{1pt}

\newcommand{\set}[1]{
  \left\{ #1 \right\}
}

%\pagenumbering{gobble} -- Este comando
%                       -- quita el número de página.
\footskip = 50pt
\renewcommand{\headrulewidth}{1pt}

\pagestyle{fancyplain}

%% Bibliografía APA
\usepackage[backend=biber]{biblatex}
\bibliography{./Bibliografia/BaseDatos}

\begin{document}
\title{UNIVERSIDAD AUT\'ONOMA DE M\'EXICO\\ Facultad de Ciencias}
\author{Autor: Adri\'an Aguilera Moreno}
\date{}
\maketitle
\begin{center}
  \includegraphics[scale=0.20]{../Imagen/Portada.jpg}\\[0.4cm]
  \Large
  \bf{Lógica Computacional}
  \normalsize
\end{center}
\newpage
\fancyhead[r]{ Lógica Computacional 2022-2}
%%%%%%%%%%%%%%%%%%%%%%%%%%%%%%%%%%%%%%%%%%%%%%%%%%%%%
\section*{\LARGE{Semanal 1}}
\begin{enumerate}
\item Escribe un ensayo sobre qué crees que es la Lógica Computacional
  y cómo te servirá a lo largo de la carrera y/o en el ámbito laboral.
  
  Desde que estudié los cursos como \textit{Estructuras Discretas} y
  \textit{Álgebra Superior I}, puede notar que la lógica (en general)
  es la que nos ayuda a realizar demostraciones, y algunos procesos
  que se basan en las ideas de demostrar preposiciones, otro clásico
  sería la recursión (explicado en ocaciones como procesos inductivos
  que basan en el teorema de la recursión). En la carrera, estas
  materias me han ayudado en gran parte a no morir en otras como;
  \textit{Gráficas y Juegos}, \textit{Álgebra Superior II},
  \textit{Álgebra Lineal I}, \textit{Probabilidad I}, etc. La lógica
  matemática fundamenta gran parte de la teoría detr\'as de la computación,
  un ejemplo de esto fue demostrar algoritmos, para estos realmente
  se busca hacer una inducción, ya sea bajo la base (definición) recursiva
  o con el invariante de ciclo en el caso iterativo, los mismo pasa
  cuando se buscan algunas características en gráficas, para esto debemos
  tener claras las definiciones de tablas de verdad, y los operadores
  binarios, de otra manera estaría muy manchado intentar demostrar
  proposiciones (o lo que busque demostrar).
  
  ``El trabajo matemático exige razonar y argumentar en forma válida
  acerca de hechos generalmente abstractos. Para ayudarnos en esta
  tarea necesitamos eliminar las ambigüedades del lenguaje ordinario
  introduciendo símbolos y conectivos cuyo uso adecuado aportará claridad
  y presición''~\cite{CA}. Buena parte de la carrera en Ciencias de la
  Computación se basa en adquirir formalismos matemáticos, en gran parte
  la lógica computacional se basará en su sintaxis, esto es el referente
  justo en el que buscaremos expresarnos con claridad.
  
    ``La lógica computacional es la misma lógica matemática aplicada a un
  contexto de las ciencias de la computación. Su uso es fundamental a
  varios niveles como en los circuitos computacionales, en la programación
  lógica y en el análisis de algoritmos''~\cite{TPERAZA}. De esta manera
  durante la carrera, al menos, se empleará la lógica computacional en materias
  como lo son; \textit{Organización y Aqruitectura de Computadoras},
  \textit{Análisis de algoritmos}, \textit{Lengiajes de programación} (respecto
  al paradigma lógico), y en general en las materias de computación teórica.
  Sin embargo no sólo nos limitaremos al uso la lógica computacional para la
  parte de computación teórica (que no siempre es tan teórica), pues programar
  de manera lógica nos aporta un plus, un ejemplo de esto sería el poder
  definir un método (en Java por ejemplo) que búsque un nodo en un árbol
  binario ordenado, entonces podemos definir este método de manera iterativa
  o implementarlo con un método auxiliar (que haga la recursión) para
  de esta  manera pasar a código casi, casi la definición de árbol binario
  ordenado (esto no necesariamente es lo más eficiente, sin embargo se puede
  implementar así).
  
  Recuerdo que en \textit{Estructuras Discretas} vimos algo de lógica de
  predicados, entonces podiamos analizar párrafos enteros al dividirlos
  en proposiciones atómicas y de esta manera encontrar si la combinación
  de estas era una tautología (en ese caso el predicado era correcto),
  una contingencia (en este caso el predicado se ve condicionado a algunos
  valores de verdad) o era una contradicción (en este caso el predicado
  siempre es incorrecto). Así, la lógica nos regala una manera distinta
  de analizar textos, y en partícular, de no caer en ambigüedades del
  lenguaje.
  
  ``Un estilo de programación radicalmente distinto al imperativo es el
  funcional. A riesgo de simplificar, la idea básica es que en lugar de
  programas con instrucciones detalladas a la computadora, se deben usar
  definiciones de funciones y aplicaciones de éstas a argumentos
  dados''~\cite{FHQ}. Es en este tipo de programación donde se hace uso
  en gran medida de la lógica computacional, si no hay buenas bases, tal
  vez no se logre entender porque una línea particular del código funciona.
  En este caso es claro que no solo estamos involucrando lo académico,
  pues lenguajes puramente funcionales como \textit{Haskell} se emplean
  hoy en día en la industria; en criptoánalisis, desarrollo web, compiladores,
  etc. Luego, existen otros con multiparadigma como \textit{Phyton} que
  es muy popular por la simpleza de su sintaxis. Es aquí cuando podemos
  notar la importancia de la lógica computacional en miras del futuro
  después de la carrera.
  
  En conclusión, siempre estamos utilizando la lógica matemática (aunque
  sea de una manera intuitiva), en nuestro caso partícular (como Científicos
  de la Computación), la usamos a diario, desde abstraer una definición
  hasta realizar la implementación de algún algoritmo.
  
  Con toda certeza sé que no abarco todo lo que se puede hacer con la
  lógica computacional, sin embargo esto es parte de lo que hoy conozco
  al respecto, tengo grandes espectativas del curso y espero pronto
  conocer muchas otras aplicaciones en el mundo académico y laboral.
  
  \\
  
  \printbibliography
\end{enumerate}
\end{document}
