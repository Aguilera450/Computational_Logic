\documentclass{article}

% Symbols
\usepackage{amsfonts, amsthm}
\usepackage{upgreek}
\usepackage{physics}
\usepackage{cancel}
\usepackage{amssymb, latexsym, amsmath}

%Algorithms
\usepackage[ruled,lined,linesnumbered,commentsnumbered]{algorithm2e}

%% Identación
\setlength{\parindent}{0cm}

% Código
\newcommand{\code}[1]{\textcolor{white!25!black}{\texttt{#1}}}
\usepackage{listings}

%AMS
\usepackage{amsthm}
\newtheorem{algo-thm}{Algoritmo}

% Proof
\renewcommand*{\proofname}{\textbf{Demostraci\'on:}}

% Theorem
\newtheorem*{theorem}{Teorema}

% Hipervínculos:
\usepackage{hyperref}

% Graphics
\usepackage{graphicx}
\usepackage{pgf}

% Color a letras.
%\usepackage[usenames,dvipsnames,svgnames,table]{xcolor}

% Tikz
\usepackage{tkz-graph}
\usepackage{tikz}
\usetikzlibrary{arrows,automata}
\usepackage{tikz}
\usetikzlibrary{arrows,automata}
%\usetikzlibrary[topaths]

% Def. Dr. César.
\usetikzlibrary{shapes,calc}
\tikzstyle{edge}=[shorten <=2pt, shorten >=2pt, >=stealth, line width=1.1pt]
\tikzstyle{blueE}=[shorten <=2pt, shorten >=2pt, >=stealth, line width=1.5pt, blue]
\tikzstyle{blackV}=[circle, fill=black, minimum size=6pt, inner sep=0pt, outer sep=0pt]
\tikzstyle{blueV}=[circle, fill=blue, draw, minimum size=6pt, line width=0.75pt, inner sep=0pt, outer sep=0pt]
\tikzstyle{redV}=[circle, fill=red, draw, minimum size=6pt, line width=0.75pt, inner sep=0pt, outer sep=0pt]
\tikzstyle{redSV}=[semicircle, fill=red, minimum size=3pt, inner sep=0pt, outer sep=0pt, rotate=225]
\tikzstyle{blueSV}=[semicircle, fill=blue, minimum size=3pt, inner sep=0pt, outer sep=0pt, rotate=225]
\tikzstyle{blackSV}=[semicircle, fill=black, minimum size=3pt, inner sep=0pt, outer sep=0pt, rotate=225]
\tikzstyle{vertex}=[circle, draw, minimum size=6pt, line width=0.75pt, inner sep=0pt, outer sep=0pt]

% Margins
\addtolength{\voffset}{-1.5cm}
\addtolength{\hoffset}{-1.5cm}
\addtolength{\textwidth}{3cm}
\addtolength{\textheight}{3cm}

%Header-Footer
\usepackage{fancyhdr}
\renewcommand{\headrulewidth}{1pt}

\newcommand{\set}[1]{
  \left\{ #1 \right\}
}

%\pagenumbering{gobble} -- Este comando
%                       -- quita el número de página.
\footskip = 50pt
\renewcommand{\headrulewidth}{1pt}

\pagestyle{fancyplain}

%% Bibliografía APA
\usepackage[backend=biber]{biblatex}
\bibliography{./Bibliografia/BaseDatos}

\begin{document}
\title{UNIVERSIDAD NACIONAL AUT\'ONOMA DE M\'EXICO\\ Facultad de Ciencias}
\author{Autor: Adri\'an Aguilera Moreno}

\date{}
\maketitle
\begin{center}
  \includegraphics[scale=0.20]{../Imagen/Portada.jpg}\\[0.4cm]
  \Large
  \bf{Lógica Computacional}
  \normalsize
\end{center}
\newpage
\fancyhead[r]{ Lógica Computacional 2022-2}
%%%%%%%%%%%%%%%%%%%%%%%%%%%%%%%%%%%%%%%%%%%%%%%%%%%%%
\section*{\LARGE{Semanal 10}}
\textbf{Instrucciones:} Realiza una investigación a cerca de los sistemas de tipos para lenguajes de programación
e incluye un resumen de la plática de este jueves 26 de mayo
``\href{https://www.facebook.com/HablandoDeMatematicas/videos/585456159435284/}{Asistentes de prueba y verificación formal}''
que se transmitirá a las 4pm por Facebook Live en: \href{https://www.facebook.com/HablandoDeMatematicas}{@HablandoDeMatematicas}.
\subsection*{Sistemas de tipos}
``Un sistema de tipos dota a los lenguajes de la capacidad de restringir los objetos
que pueden ser asignados a las variables. Esto permite una cierta potencia a la hora de
detectar errores y mejora la comprensión del código''. En general, los facilita el trabajo
de los compiladores y ayuda a depurar errores.

``A los lenguajes de programación le sirve la información de tipos para poder
atajar posibles errores antes de que ocurran, estos son los chequeos estáticos``.
Estos errores no son lógicos, pues estos no son tan triviales de reconocer y es
complicado reconocerlos solo con un lenguaje tipado.

Existen dos grupos de tipos importantes para los lenguajes de programación, estos
son, de tipo \textbf{Church}\footnote{Se trata de cualquier tipo explícito.} y los
tipos \textbf{Curry}\footnote{Son sistemas implícitos.}.

``En el contexto de la programación declarativa, un sistema de tipos consiste en la
teoría base que permite asociar a cada término de un lenguaje declarativo un tipo.''
Por ejemplo, en \code{Haskell} tenemos algunos tipos como \code{Char}, \code{Bool},
etc.

Algunos problemas que resulven los sistemas de tipos son:
\newcommand{\localtextbulletone}{\textcolor{black}{\raisebox{.45ex}{\rule{.6ex}{.6ex}}}}
\renewcommand{\labelitemi}{\localtextbulletone}
\begin{itemize}
\item \textbf{Comprobación de tipos}, que es comprobar que un elemento es de algún tipo
  en partícular.
\item \textbf{Tipificación}, esto es, identificar el tipo que satisface un modelo para
  el programa.
\item \textbf{Habitabilidad}, consiste en comprobar si existe un elemento tipificable
  en el lenguaje en uso.
\end{itemize}
Estas son algunas de las funciones que tiene el tipado de lenguajes.

\subsection*{Asistentes de prueba y verificación formal}
Basados en la lógica que satisfacen modelos, los asistentes de prueba
\\

\printbibliography
\end{document}
