\documentclass{article}

% Symbols
\usepackage[T1]{fontenc}
\usepackage{upgreek}
\usepackage{physics}
\usepackage{cancel}
\usepackage{amsfonts, amsthm}
\usepackage{amssymb, latexsym, amsmath}


% Proof
\renewcommand*{\proofname}{\textbf{Soluci\'on:}}

% Theorem
\newtheorem*{theorem}{Teorema}

%Algorithms
\usepackage[ruled,lined,linesnumbered,commentsnumbered]{algorithm2e}

%% Identación
\setlength{\parindent}{0cm}

% Código
\newcommand{\code}[1]{\textcolor{white!25!black}{\texttt{#1}}}
\usepackage{listings}

%AMS
\usepackage{amsthm}
\newtheorem{algo-thm}{Algoritmo}

% Graphics
\usepackage{graphicx}
\usepackage{pgf}

% Margins
\addtolength{\voffset}{-1.5cm}
\addtolength{\hoffset}{-1.5cm}
\addtolength{\textwidth}{3cm}
\addtolength{\textheight}{3cm}

%Header-Footer
\usepackage{fancyhdr}
\renewcommand{\headrulewidth}{1pt}

\newcommand{\set}[1]{
  \left\{ #1 \right\}
}

\footskip = 50pt
\renewcommand{\headrulewidth}{1pt}

\pagestyle{fancyplain}

\begin{document}
\title{UNIVERSIDAD NACIONAL AUT\'ONOMA DE M\'EXICO\\ Facultad de Ciencias}
\author{Autor: Adri\'an Aguilera Moreno}
\date{}
\maketitle
\begin{center}
  \includegraphics[scale=0.20]{../Imagen/Portada.jpg}\\[0.4cm]
  \Large
  \bf{Lógica Computacional}
  \normalsize
\end{center}
\newpage
\fancyhead[r]{ Lógica Computacional 2022-2}
%%%%%%%%%%%%%%%%%%%%%%%%%%%%%%%%%%%%%%%%%%%%%%%%%%%%%
\section*{\LARGE{Semanal 8}}
Para cada uno de los siguientes ejercicios, \textbf{justifica ampliamente} tu respuesta.
\\

1. \textbf{Obtén} la Forma Normal Negativa y la Forma Normal Prenex de las siguientes fórmulas:
\newcommand{\localtextbulletone}{\textcolor{black}{\raisebox{.45ex}{\rule{.6ex}{.6ex}}}}
\renewcommand{\labelitemi}{\localtextbulletone}
\begin{itemize}
\item $\forall_{x}\forall_{y}\left(\exists_{z}(R(x,z) \rightarrow R(z,y)) \rightarrow P(y,x)\right)
  \rightarrow \exists_{x}\left(\forall_{y}P(x,y) \rightarrow \exists_{y}P(y,x)\right)
  \rightarrow \forall_{x}P(x,x)$:
  \begin{proof}
    Llamemos $\varphi$ a la ecuación anterior. Así, al rectificar nuestra ecuación tenemos que
    \begin{eqnarray*}
      rec(\varphi) &=& \forall_{x'}\forall_{y'}\left(\exists_{z'}(R(x',z') \rightarrow R(z',y')) \rightarrow P(y',x')\right)
      \rightarrow \exists_{m}\left(\forall_{n}P(m,n)  \rightarrow \exists_{y}P(y,m)\right)\\
      & &\rightarrow \forall_{x}P(x,x)\\
      &=& \varphi'
    \end{eqnarray*}
    encontrando la Forma Normal Negativa tenemos que
    \begin{eqnarray*}
      fnn(\varphi') &=& \forall_{x'}\forall_{y'}\left(\exists_{z'}(\neg R(x',z') \lor  R(z',y')) \rightarrow  P(y',x')\right)
      \rightarrow \exists_{m}\left(\neg \forall_{n}P(m,n) \lor \exists_{y}P(y,m)\right)\\
      & & \rightarrow \forall_{x}P(x,x)\\
      &=& \forall_{x'}\forall_{y'}\left(\neg\exists_{z'}(\neg R(x',z') \lor  R(z',y')) \lor  P(y',x')\right)
      \rightarrow \exists_{m}\left(\neg \forall_{n}P(m,n) \lor \exists_{y}P(y,m)\right)\\
      & & \rightarrow \forall_{x}P(x,x)\\
      &=& \neg \forall_{x'}\forall_{y'}\left(\neg\exists_{z'}(\neg R(x',z') \lor  R(z',y')) \lor  P(y',x')\right)
      \lor \exists_{m}\left(\neg \forall_{n}P(m,n) \lor \exists_{y}P(y,m)\right)\\
      & & \rightarrow \forall_{x}P(x,x)\\
      &=& \neg (\neg \forall_{x'}\forall_{y'}\left(\neg\exists_{z'}(\neg R(x',z') \lor  R(z',y')) \lor  P(y',x')\right)
      \lor \exists_{m}\left(\neg \forall_{n}P(m,n) \lor \exists_{y}P(y,m)\right))\\
      & & \lor\; \forall_{x}P(x,x)\\
      &=& \forall_{x'}\forall_{y'}\left(\neg\exists_{z'}(\neg R(x',z') \lor  R(z',y')) \lor  P(y',x')\right)
      \land \neg \exists_{m}\left(\neg \forall_{n}P(m,n) \lor \exists_{y}P(y,m)\right)\\
      & & \lor\; \forall_{x}P(x,x)\\
      &=& \forall_{x'}\forall_{y'}\left(\forall_{z'}(R(x',z') \land \neg R(z',y')) \lor  P(y',x')\right)
      \land \forall_{m}\left(\forall_{n} P(m,n) \land \forall_{y} \neg P(y,m)\right)\\
      & & \lor\; \forall_{x}P(x,x)\\
      &=& \varphi''
    \end{eqnarray*}
    por último, encontremos la Forma Normal de Prenex. Esto es
    \begin{eqnarray*}
      prenex(\varphi'') &=& \forall_{x'}\forall_{y'}\left(\forall_{z'}(R(x',z') \land \neg R(z',y')) \lor  P(y',x')\right)
      \land \forall_{m}\left(\forall_{n} P(m,n) \land \forall_{y} \neg P(y,m)\right)\\
      & & \lor\; \forall_{x}P(x,x)\\
      &=& \forall_{x}(\forall_{x'}\forall_{y'}\left(\forall_{z'}(R(x',z') \land \neg R(z',y')) \lor  P(y',x')\right)
      \land \forall_{m}\left(\forall_{n} P(m,n) \land \forall_{y} \neg P(y,m)\right)\\
      & & \lor\; P(x,x))\\
      &=& \forall_{x}(\forall_{x'}\forall_{y'}\left(\forall_{z'}(R(x',z') \land \neg R(z',y')) \lor  P(y',x')\right)
      \land \forall_{y}\forall_{n}\forall_{m}\left(P(m,n) \land \neg P(y,m)\right)\\
      & & \lor\; P(x,x))\\
      &=& \forall_{x}(\forall_{y}\forall_{n}\forall_{m}\forall_{x'}\forall_{y'}\left(\forall_{z'}(R(x',z') \land \neg R(z',y')) \lor  P(y',x')\right)
      \land \left(P(m,n) \land \neg P(y,m)\right)\\
      & & \lor\; P(x,x))\\
      &=& \forall_{x}(\forall_{z'}\forall_{y}\forall_{n}\forall_{m}\forall_{x'}\forall_{y'}\left((R(x',z') \land \neg R(z',y')) \lor  P(y',x')\right)
      \land \left(P(m,n) \land \neg P(y,m)\right)\\
      & & \lor\; P(x,x))\\
      &=& \forall_{x}\forall_{z'}\forall_{y}\forall_{n}\forall_{m}\forall_{x'}\forall_{y'}\left(((R(x',z') \land \neg R(z',y')) \lor  P(y',x')\right)
      \land P(m,n) \land \neg P(y,m))\\
      & & \lor\; P(x,x))
    \end{eqnarray*}
  \end{proof}
\item $\forall_{x}\forall_{y}\left(P(x,y) \rightarrow \exists_{z}(R(x,y) \land R(z,y))\right)
  \rightarrow \exists_{x}\left(\forall_{y}P(x,y) \land \forall_{y}P(y,x)\right) \lor \exists_{x}P(x,x)$:
  \begin{proof}
    Llamemos $\phi$ a la ecuación anterior. Así, al rectificar nuestra ecuación tenemos que
    \begin{eqnarray*}
      rec(\phi) &=& \forall_{x'}\forall_{y'}\left(P(x',y') \rightarrow \exists_{z}(R(x',y') \land R(z,y'))\right)
      \rightarrow \exists_{m}\left(\forall_{n}P(m,n) \land \forall_{s}P(s,m)\right) \lor \exists_{x}P(x,x)\\
      &=& \phi'
    \end{eqnarray*}
    encontrando la Forma Normal Negativa tenemos que
    \begin{eqnarray*}
      fnn(\phi') &=& \forall_{x'}\forall_{y'}\left(\neg P(x',y') \lor \exists_{z}(R(x',y') \land R(z,y'))\right)
      \rightarrow \exists_{m}\left(\forall_{n}P(m,n) \land \forall_{s}P(s,m)\right) \lor \exists_{x}P(x,x)\\
      &=& \neg \forall_{x'}\forall_{y'}\left(\neg P(x',y') \lor \exists_{z}(R(x',y') \land R(z,y'))\right)
      \lor \exists_{m}\left(\forall_{n}P(m,n) \land \forall_{s}P(s,m)\right) \lor \exists_{x}P(x,x)\\
      &=& \exists_{x'}\exists_{y'}\left(P(x',y') \land \forall_{z}(\neg R(x',y') \lor \neg R(z,y'))\right)
      \lor \exists_{m}\left(\forall_{n}P(m,n) \land \forall_{s}P(s,m)\right) \lor \exists_{x}P(x,x)\\
      &=& \phi''
    \end{eqnarray*}
    por último, encontremos la Forma Normal de Prennex, esto es
    \begin{eqnarray*}
      prennex(\phi'') &=& \exists_{x'}\exists_{y'}\forall_{z}\left(P(x',y') \land (\neg R(x',y') \lor \neg R(z,y'))\right)
      \lor \exists_{m}\forall_{n}\forall_{s}\left(P(m,n) \land P(s,m)\right) \lor \exists_{x}P(x,x)\\
      &=& \exists_{x'}\exists_{y'}\forall_{z}\exists_{x}\exists_{m}\forall_{n}\forall_{s}\left(P(x',y')
      \land (\neg R(x',y') \lor \neg R(z,y')) \lor \left(P(m,n) \land P(s,m)\right) \lor P(x,x)\right)
    \end{eqnarray*}
  \end{proof}
\item $\forall_{x}\exists_{y}\left(B(x, a) \land M (y, x)\right) \rightarrow \exists_{z}\left(B(x, w) \land M (z, x)\right)$:
  \begin{proof}
    Llamemos $\tqheta$ a la ecuación anterior. Así, al rectificar nuestra ecuación tenemos que que
    \begin{eqnarray*}
      rec(\theta) &=& \forall_{x'}\exists_{y'}\left(B(x', a) \land M(y', x')\right)
      \rightarrow \exists_{z}\left(B(x, w) \land M (z, x)\right)\\
      &=& \theta'
    \end{eqnarrqay*}
    encontrando la Forma Normal Negativa, tenemos que
    \begin{eqnarray*}
      fnn(\theta') &=& \neg \forall_{x'}\exists_{y'}\left(B(x', a) \land M(y', x')\right)
      \lor \exists_{z}\left(B(x, w) \land M (z, x)\right)\\
      &=& \exists_{x'}\forall_{y'}\left(\neg B(x', a) \lor \neg M(y', x')\right)
      \lor \exists_{z}\left(B(x, w) \land M (z, x)\right)\\
      &=& \theta''
    \end{eqnarray*}
    por último, encontremos la Forma Normal de Prennex, esto es
    \begin{eqnarray*}
      prennex(\theta'') &=& \exists_{z}\exists_{x'}\forall_{y'}\left(\neg B(x', a) \lor \neg M(y', x')\right)
      \lor \left(B(x, w) \land M (z, x)\right)
    \end{eqnarray*}
  \end{proof}
\end{itemize}
\end{document}
