\documentclass{article}

% Symbols
\usepackage{amsfonts, amsthm}
\usepackage{upgreek}
\usepackage{physics}
\usepackage{cancel}
\usepackage{amssymb, latexsym, amsmath}
\usepackage{xcolor}

%Algorithms
\usepackage[ruled,lined,linesnumbered,commentsnumbered]{algorithm2e}

%% Identación
\setlength{\parindent}{0cm}

% Código
\newcommand{\code}[1]{\textcolor{white!25!black}{\texttt{#1}}}
\usepackage{listings}

%AMS
\usepackage{amsthm}
\newtheorem{algo-thm}{Algoritmo}

% Proof
\renewcommand*{\proofname}{\textbf{Soluci\'on:}}
% Theorem
\newtheorem*{theorem}{Teorema}

% Graphics
\usepackage{graphicx}
\usepackage{pgf}

% Color a letras.
%\usepackage[usenames,dvipsnames,svgnames,table]{xcolor}

% << >>
\usepackage[T1]{fontenc}

% Tikz
\usepackage{tkz-graph}
\usepackage{tikz}
\usetikzlibrary{arrows,automata}
\usepackage{tikz}
\usetikzlibrary{arrows,automata}
%\usetikzlibrary[topaths]

% Def. Dr. César.
\usetikzlibrary{shapes,calc}
\tikzstyle{edge}=[shorten <=2pt, shorten >=2pt, >=stealth, line width=1.1pt]
\tikzstyle{blueE}=[shorten <=2pt, shorten >=2pt, >=stealth, line width=1.5pt, blue]
\tikzstyle{blackV}=[circle, fill=black, minimum size=6pt, inner sep=0pt, outer sep=0pt]
\tikzstyle{blueV}=[circle, fill=blue, draw, minimum size=6pt, line width=0.75pt, inner sep=0pt, outer sep=0pt]
\tikzstyle{redV}=[circle, fill=red, draw, minimum size=6pt, line width=0.75pt, inner sep=0pt, outer sep=0pt]
\tikzstyle{redSV}=[semicircle, fill=red, minimum size=3pt, inner sep=0pt, outer sep=0pt, rotate=225]
\tikzstyle{blueSV}=[semicircle, fill=blue, minimum size=3pt, inner sep=0pt, outer sep=0pt, rotate=225]
\tikzstyle{blackSV}=[semicircle, fill=black, minimum size=3pt, inner sep=0pt, outer sep=0pt, rotate=225]
\tikzstyle{vertex}=[circle, draw, minimum size=6pt, line width=0.75pt, inner sep=0pt, outer sep=0pt]

% Margins
\addtolength{\voffset}{-1.5cm}
\addtolength{\hoffset}{-1.5cm}
\addtolength{\textwidth}{3cm}
\addtolength{\textheight}{3cm}

%Header-Footer
\usepackage{fancyhdr}
\renewcommand{\headrulewidth}{1pt}

\newcommand{\set}[1]{
  \left\{ #1 \right\}
}

%\pagenumbering{gobble} -- Este comando
%                       -- quita el número de página.
\footskip = 50pt

\pagestyle{fancyplain}

\begin{document}
\title{UNIVERSIDAD NACIONAL AUT\'ONOMA DE M\'EXICO\\ Facultad de Ciencias}
\author{Autor: Adri\'an Aguilera Moreno}
\date{}
\maketitle
\begin{center}
  \includegraphics[scale=0.20]{../Imagen/Portada.jpg}\\[0.4cm]
  \Large
  \bf{Lógica Computacional}
  \normalsize
\end{center}
\newpage
\fancyhead[r]{ Lógica Computacional 2022-2}
%%%%%%%%%%%%%%%%%%%%%%%%%%%%%%%%%%%%%%%%%%%%%%%%%%%%%
\section*{\LARGE{Semanal 4}}
Para cada uno de los ejercicios, \textbf{justifica ampliamente} tu respuesta.
\begin{itemize}
  %%%%%%%%%%%%%%%%%%%%%%%%%%%%%%%%%%%%%%%%%%%%%%%%%%%%% Ejercicio 01.  
\item[1.] Utilizándo resolución binaria, \textbf{verifica} si los siguientes
  argumentos son verdaderos o falsos:
  \begin{itemize}
  \item[$\cdot$)] $\{p \rightarrow q, \neg (q \land \neg r), r \rightarrow s\} \models p \rightarrow s$.
  \item[$\cdot$)] Si estudias Seguridad Informática y eres una persona antisocial, entonces eres un
    Hacker. Estudias Seguridad Informática pero no te gustan los videojuegos. Así que
    no eres una persona antisocial.
  \item[$\cdot$)] $\{p \rightarrow q, r \lor s, \neg s \rightarrow \neg t, \neg q \lor s, \neg s,
    \neg p \land r \rightarrow u, w \lor r \} \models u \land w$.
  \end{itemize}
  
  \begin{proof} Abordemos el ejercicio por partes:
    \begin{enumerate}
    \item Veamos que
      \begin{eqnarray*}
        &1.& p \rightarrow q \equiv \neg p \lor q\\
        &2.& \neg (q \land \neg r) \equiv  \neg q \lor r\\
        &3.& r \rightarrow s \equiv \neg r \lor s\\
        &4.& \neg (p \rightarrow s) \equiv p \land \neg s
        \hspace*{1.3cm} \text{Se niega el consecuente.}\\
        &5.& p \hspace*{4cm} \text{por } 4.\\
        &6.& \neg s \hspace*{3.8cm} \text{por } 4.\\
        &7.& q \hspace*{4cm} \text{Res}(1,5).\\
        &8.& r \hspace*{4cm} \text{Res}(2,7).\\
        &9.& s \hspace*{4cm} \text{Res}(3,8).\\
        &10.& \square \hspace*{3.9cm} \text{Res}(6,9).
      \end{eqnarray*}
      Como concluimos que se llega a una contradicción, entonces
      podemos asegurar que el conjunto no genera como conclusión
      a $\neg (p \rightarrow s)$, esto implica que si genere a
      $p \rightarrow s$ como conclusión.
      \[
      \therefore\;\; \text{El argumento es correcto.}
      \]
    \item Definamos
      \begin{eqnarray*}
        &p\; :& \text{estudias Seguridad Informática}.\\
        &q\; :& \text{eres una persona antisocial}.\\
        &r\; :& \text{eres hacker}.\\
        &s\; :& \text{te gustan los videojuegos}.
      \end{eqnarray*}
      ahora, nuestro conjunto de proposiciones lógicas se resume en
      \begin{eqnarray*}
        \{p \land q \rightarrow r, p \land \neg s\} \models \neg q
      \end{eqnarray*}
      realizando resolución binaria tenemos que
      \begin{eqnarray*}
        &1.& p \land q \rightarrow r \equiv \neg (p \land q) \lor r \equiv \neg p \lor \neg q \lor r\\
        &2.& p \land \neg s\\
        &3.& \neg \neg q \equiv q \hspace*{5cm} \text{Se niega el consecuente.}\\
        &4.& p \hspace*{6.1cm} \text{por } 2.\\
        &5.& \neg s \hspace*{5.9cm} \text{por } 2.\\
        &6.& \neg q \lor r \hspace*{5.3cm} \text{Res}(1,4).\\
        &7.& r \hspace*{6.1cm} \text{Res}(3,6).
      \end{eqnarray*}
    \item Veamos que
      \begin{eqnarray*}
        &1.& p \rightarrow q \equiv \neg p \lor q\\
        &2.& r \lor s\\
        &3.& \neg s \rightarrow \neg t \equiv s \lor \neg t\\
        &4.& \neg q \lor s\\
        &5.& \neg s\\
        &6.& \neg p \land r \rightarrow  u \equiv \neg (\neg p \land r) \lor u \equiv p \lor \neg r \lor u\\
        &7.& w \lor t\\
        &8.& \neg (u \land w) \equiv \neg u \lor \neg w \hspace*{5cm} \text{Se niega el consecuente.}\\
        &9.& r \hspace*{8cm} \text{Res}(2,5).\\
        &10.& \neg t \hspace*{7.85cm} \text{Res}(3,5).\\
        &11.& \neg q \hspace*{7.8cm} \text{Res}(4,5).\\
        &12.& \neg p \hspace*{7.8cm} \text{Res}(1,11).\\
        &13.& p \lor u \hspace*{7.45cm} \text{Res}(6,9).\\
        &14.& \neg r \lor u \hspace*{7.3cm} \text{Res}(6,12).\\
        &15.& w \hspace*{8cm} \text{Res}(7,10).\\
        &14.& \neg u \hspace*{7.8cm} \text{Res}(8,15).\\
        &15.& u \hspace*{8.05cm}\text{Res}(9,14).\\
        &16.& \square \hspace*{8cm} \text{Res}(14,15).
      \end{eqnarray*}
      como concluimos que
      \[
      \{p \rightarrow q, r \lor s, \neg s \rightarrow \neg t, \neg q \lor s, \neg s,
      \neg p \land r \rightarrow u, w \lor r \} \cup \{\neg (u \land w)\}
      \]
      es una contradicción, entonces se sigue que
      \[
      \therefore\;\; \text{El argumento es correcto.}
      \]
    \end{enumerate}
  \end{proof}
\end{document}
