\documentclass{article}

% Symbols
\usepackage[T1]{fontenc}
\usepackage{upgreek}
\usepackage{physics}
\usepackage{cancel}
\usepackage{amsfonts, amsthm}
\usepackage{amssymb, latexsym, amsmath}


% Proof
\renewcommand*{\proofname}{\textbf{Soluci\'on:}}

% Theorem
\newtheorem*{theorem}{Teorema}

%Algorithms
\usepackage[ruled,lined,linesnumbered,commentsnumbered]{algorithm2e}

%% Identación
\setlength{\parindent}{0cm}

% Código
\newcommand{\code}[1]{\textcolor{white!25!black}{\texttt{#1}}}
\usepackage{listings}

%AMS
\usepackage{amsthm}
\newtheorem{algo-thm}{Algoritmo}

% Graphics
\usepackage{graphicx}
\usepackage{pgf}

% Margins
\addtolength{\voffset}{-1.5cm}
\addtolength{\hoffset}{-1.5cm}
\addtolength{\textwidth}{3cm}
\addtolength{\textheight}{3cm}

%Header-Footer
\usepackage{fancyhdr}
\renewcommand{\headrulewidth}{1pt}

\newcommand{\set}[1]{
  \left\{ #1 \right\}
}

\footskip = 50pt
\renewcommand{\headrulewidth}{1pt}

\pagestyle{fancyplain}

\begin{document}
\title{UNIVERSIDAD NACIONAL AUT\'ONOMA DE M\'EXICO\\ Facultad de Ciencias}
\author{Autor: Adri\'an Aguilera Moreno}
\date{}
\maketitle
\begin{center}
  \includegraphics[scale=0.20]{../Imagen/Portada.jpg}\\[0.4cm]
  \Large
  \bf{Lógica Computacional}
  \normalsize
\end{center}
\newpage
\fancyhead[r]{ Lógica Computacional 2022-2}
%%%%%%%%%%%%%%%%%%%%%%%%%%%%%%%%%%%%%%%%%%%%%%%%%%%%%
\section*{\LARGE{Semanal 5}}
Para cada uno de los ejercicios, \textbf{justifica ampliamente} tu respuesta.

\newcommand{\localtextbulletone}{\textcolor{black}{\raisebox{.45ex}{\rule{.6ex}{.6ex}}}}
\renewcommand{\labelitemi}{\localtextbulletone}
\begin{itemize}
  %%%%%%%%%%%%%%%%%%%%%%%%%%%%%%%%%%%%%%%%%%%%%%%%%%%%% Ejercicio 01.
\item \textbf{Da} la especificación formal del siguiente argumento,
  definiendo previamente una signtura adecuada.
  \begin{center}
    \guillemotleft Los alumnos de la Facultad de Ciencias y los
    programadores sólo alcanzan su máximo nivel cuando la luna
    es rosa.\guillemotright
  \end{center}
  
  \begin{proof}
    Definamos la signatura para este enunciado, esto es
    \[
    Personas
    \cup \{luna\}
    \cup \{colorear^{2}\}
    \cup \{Pers^{1}, AlumnoFC^{1}, Prog^{1}, EsColor^{2}, MaxNivel^{1}\}
    \]
    que significan:
    \begin{itemize}
    \item $colorear(x, y) = x$ coloreado de $y$ [se devuelve $x$, solo cambia su estado],
      esto es una función.
    \item $Pers(x) = x$ es una persona.
    \item $AlumnoFC(x) = x$ es alumno de la Facultad de Ciencias.
    \item $Prog(x) = x$ es programador.
    \item $EsColor(x, y) = x$ es de color $y$.
    \item $MaxNivel(x) = x$ ha alcanzado su máximo nivel.
    \end{itemize}
    Dada la signatura anterior, tenemos que el enunciado se traduce como
    \begin{eqnarray*}
      \forall_{x} \left(Pers(x) \land AlumnoFC(x) \land Prog(x)
      \rightarrow \left(EsColor(colorear(luna, rosa), rosa) \rightarrow MaxNivel(x)\right)\right)
    \end{eqnarray*}
  \end{proof}
  %%%%%%%%%%%%%%%%%%%%%%%%%%%%%%%%%%%%%%%%%%%%%%%%%%%%% Ejercicio 02.
\item Dadas las siguientes expresiones:
  \begin{enumerate}
  \item $\forall_{x}\forall_{u}\left(M(a, x) \rightarrow M(a, u)\right)$
  \item $(\forall_{x, u}\left(M(a, x)\right) \rightarrow M(a, u)$
  \item $\forall_{x}(M(a, x) \rightarrow M(a, u))$
  \end{enumerate}
  \textbf{Realiza} lo siguiente:
  \begin{itemize}
  \item[$a$)] Indica si las relaciones
    \begin{center}
      \begin{tabular}{ c  c  c}
        $(1) \sim_{\alpha} (2)$ \hspace*{1cm} & $(2) \sim_{\alpha} (3)$ \hspace*{1cm} & $(1) \sim_{\alpha} (3)$
      \end{tabular} 
    \end{center}
    de $\alpha$-equivalencia se cumplen.
  \item[$b$)] Aplica la sustitución
    \[
    \delta = [x, u := \ell(a, n, d, r, o), c(i, z, o)]
    \]
    a cada una de las expresiones.
  \end{itemize}
\end{itemize}
\begin{proof}
  Primero, encontremos las $\sim_{\alpha}$ equivalencias, esto es
  \begin{eqnarray*}
    1.\; \forall_{x}\forall_{u}\left(M(a,x) \rightarrow M(a,u)\right)
    &\sim_{\alpha}& \forall_{w}\forall_{u}(M(a, w) \rightarrow M(a, u))\\
    &\sim_{\alpha}& \forall_{w}\forall_{v}(M(a, w) \rightarrow M(a, v))
  \end{eqnarray*}
  
  \begin{eqnarray*}
    2.\; (\forall_{xu}M(a,x)) \rightarrow M(a,u)
    &\sim_{\alpha}& (\forall_{wu}M(a, w)) \rightarrow M(a, u)\\
    &\sim_{\alpha}& (\forall_{wv}M(a, w)) \rightarrow M(a, u)\\
    &\sim_{\alpha}& (\forall_{w}M(a, w)) \rightarrow M(a, u)
  \end{eqnarray*}
  
  \begin{eqnarray*}
    3.\; \forall_{x}\left(M(a,x) \rightarrow M(a,u)\right)
    \sim_{\alpha} \forall_{w}\left(M(a,w) \rightarrow M(a,u)\right)
  \end{eqnarray*}
  Ahora, veamos que $(1) \not\sim_{\alpha} (2)$, pues difieren en
  variables no ligadas, como lo es $u$ en $(2)$, pues en $(1)$ la
  única variable no ligada es $a$.
  
  En cambio, $(2) \sim_{\alpha} (3)$, pues por definición de $\sim_{\alpha}$
  se tiene que a lo más difieran en sus variables ligadas, el cual es el caso,
  ya que, $\{a, u\}$ son variables no ligadas.
  
  Como $(1) \not\sim_{\alpha} (2)$ y $(2) \sim_{\alpha} (3)$, entonces por
  transitividad tenemos que $1 \not \sim_{\alpha} (3)$.
  
  Ahora, realicemos las sustituciones con $\delta$, estas las aplicaremos a
  las $\sim_{\alpha}$ a las que hemos llegado, esto es
  
  \begin{eqnarray*}
    1.\; \underbrace{\forall_{w}\forall_{v}(M(a, w) \rightarrow M(a, v))}_{\delta}
    &=& \forall_{w}\underbrace{\forall_{v}(M(a, w) \rightarrow M(a, v))}_{\delta}\\
    &=& \forall_{w}\forall_{v}(\underbrace{M(a, w) \rightarrow M(a, v)}_{\delta})\\
    &=& \forall_{w}\forall_{v}(\underbrace{M(a, w)}_{\delta} \rightarrow \underbrace{M(a, v)}_{\delta})\\
    &=& \forall_{w}\forall_{v}(M(\underbrace{a}_{\delta}, \underbrace{w}_{\delta}) \rightarrow M(\underbrace{a}_{\delta}, \underbrace{v}_{\delta}))\\
    &=& \forall_{w}\forall_{v}(M(a, w) \rightarrow M(a, v))
  \end{eqnarray*}
  
  \begin{eqnarray*}
    2.\; \underbrace{(\forall_{w}M(a, w)) \rightarrow M(a, u)}_{\delta}
    &=& (\underbrace{\forall_{w}M(a, w)}_{\delta}) \rightarrow \underbrace{M(a, u)}_{\delta}\\
    &=& (\forall_{w}M(\underbrace{a}_{\delta}, \underbrace{w}_{\delta})) \rightarrow M(\underbrace{a}_{\delta}, \underbrace{u}_{\delta})\\
    &=& (\forall_{w}M(a, w)) \rightarrow M(a, c(i, z, o))
  \end{eqnarray*}
  
  \begin{eqnarray*}
    3.\; \underbrace{\forall_{w}\left(M(a,w) \rightarrow M(a,u)\right)}_{\delta}
    &=& \forall_{w}\underbrace{\left(M(a,w) \rightarrow M(a,u)\right)}_{\delta}\\
    &=& \forall_{w}(\underbrace{M(a,w)}_{\delta} \rightarrow \underbrace{M(a,u)}_{\delta})\\
    &=& \forall_{w}(M(\underbrace{a}_{\delta},\underbrace{w}_{\delta}) \rightarrow M(\underbrace{a}_{\delta},\underbrace{u}_{\delta}))\\
    &=& \forall_{w}\left(M(a,w) \rightarrow M(a,c(i,z,o))\right)
  \end{eqnarray*}
\end{proof}
\end{document}
