\documentclass{article}

% Symbols
\usepackage{amsfonts, amsthm}
\usepackage{upgreek}
\usepackage{physics}
\usepackage{cancel}
\usepackage{amssymb, latexsym, amsmath}

%Algorithms
\usepackage[ruled,lined,linesnumbered,commentsnumbered]{algorithm2e}

%% Identación
\setlength{\parindent}{0cm}

% Código
\newcommand{\code}[1]{\textcolor{white!25!black}{\texttt{#1}}}
\usepackage{listings}

%AMS
\usepackage{amsthm}
\newtheorem{algo-thm}{Algoritmo}

% Proof
\renewcommand*{\proofname}{\textbf{Demostraci\'on:}}
% Theorem
\newtheorem*{theorem}{Teorema}

% Graphics
\usepackage{graphicx}
\usepackage{pgf}

% Color a letras.
%\usepackage[usenames,dvipsnames,svgnames,table]{xcolor}

% Tikz
\usepackage{tkz-graph}
\usepackage{tikz}
\usetikzlibrary{arrows,automata}
\usepackage{tikz}
\usetikzlibrary{arrows,automata}
%\usetikzlibrary[topaths]

% Def. Dr. César.
\usetikzlibrary{shapes,calc}
\tikzstyle{edge}=[shorten <=2pt, shorten >=2pt, >=stealth, line width=1.1pt]
\tikzstyle{blueE}=[shorten <=2pt, shorten >=2pt, >=stealth, line width=1.5pt, blue]
\tikzstyle{blackV}=[circle, fill=black, minimum size=6pt, inner sep=0pt, outer sep=0pt]
\tikzstyle{blueV}=[circle, fill=blue, draw, minimum size=6pt, line width=0.75pt, inner sep=0pt, outer sep=0pt]
\tikzstyle{redV}=[circle, fill=red, draw, minimum size=6pt, line width=0.75pt, inner sep=0pt, outer sep=0pt]
\tikzstyle{redSV}=[semicircle, fill=red, minimum size=3pt, inner sep=0pt, outer sep=0pt, rotate=225]
\tikzstyle{blueSV}=[semicircle, fill=blue, minimum size=3pt, inner sep=0pt, outer sep=0pt, rotate=225]
\tikzstyle{blackSV}=[semicircle, fill=black, minimum size=3pt, inner sep=0pt, outer sep=0pt, rotate=225]
\tikzstyle{vertex}=[circle, draw, minimum size=6pt, line width=0.75pt, inner sep=0pt, outer sep=0pt]

% Margins
\addtolength{\voffset}{-1.5cm}
\addtolength{\hoffset}{-1.5cm}
\addtolength{\textwidth}{3cm}
\addtolength{\textheight}{3cm}

%Header-Footer
\usepackage{fancyhdr}
\renewcommand{\headrulewidth}{1pt}

\newcommand{\set}[1]{
  \left\{ #1 \right\}
}

%\pagenumbering{gobble} -- Este comando
%                       -- quita el número de página.
\footskip = 50pt
\renewcommand{\headrulewidth}{1pt}

\pagestyle{fancyplain}

\begin{document}
\title{UNIVERSIDAD AUT\'ONOMA DE M\'EXICO\\ Facultad de Ciencias}
\author{Autor: Adri\'an Aguilera Moreno}
\date{}
\maketitle
\begin{center}
  \includegraphics[scale=0.20]{../Imagen/Portada.jpg}\\[0.4cm]
  \Large
  \bf{Lógica Computacional}
  \\

  \normalsize
  \begin{center}
    \fbox{
      \begin{minipage}[b][1\height]%
        [t]{0.867\textwidth}
        La propuesta de estos boletines fue hecha por:
        \begin{itemize}
        \item Dr. Favio E. Miranda Parea.
        \item Dra. Lourdes González Huesca.
        \item Mtra. A. Liliana Reyes Cabello.
        \end{itemize}
    \end{minipage}}
  \end{center}
\end{center}
\newpage
\fancyhead[r]{ Lógica Computacional 2022-2}
%%%%%%%%%%%%%%%%%%%%%%%%%%%%%%%%%%%%%%%%%%%%%%%%%%%%%
\section*{\LARGE{Boletín 1}}
\begin{enumerate}
  %%%%%%%%%%%%%%%%%%%%%%%%%%%% Ejercicio 01  %%%%%%%%%%%%%%%%%%%%%%%%%%%%
\item Elimina los paréntesis innecesarios en las siguientes expresiones:
  \begin{itemize}
  \item[$a$)] $\left((p \lor q) \rightarrow r\right) \leftrightarrow \left((\neg r) \rightarrow (\neg (p \lor q))\right)$
  \item[$b$)] $\neg \left(((p \land (p \rightarrow (\neg p))) \land q) \land p\right)$
  \item[$c$)] $\left(p \rightarrow (q \land (\neg q)) \right) \rightarrow \left((\neg p) \rightarrow p\right)$
  \item[$d$)] $(\neg s) \rightarrow \left((\neg t) \land \neg (p \lor q)\right)$
  \end{itemize}
  
  $\triangledown$ \textbf{Solución:}
  
  \begin{eqnarray*}
    &(a)& p \lor q \rightarrow r \leftrightarrow \neg r \rightarrow \neg (p \lor q)\\
    &(b)& \neg (p \land (p \rightarrow \neg q) \land q \land p)\\
    &(c)& (p \rightarrow q \land \neg q) \rightarrow (\neg p \rightarrow p)\\
    &(d)& \neg s \rightarrow \neg t \land \neg (p \lor q)
  \end{eqnarray*}
  \hfill $\lhd$
  %%%%%%%%%%%%%%%%%%%%%%%%%%%% Ejercicio 02  %%%%%%%%%%%%%%%%%%%%%%%%%%%%
\item Definie las siguientes funciones, indicando claramente en cada caso el dominio
  y contradominio de la función definida:
  \begin{itemize}
  \item[$a$)] Profundidad de una fórmula: \textit{depth}$(\varphi)$ devuelve la
    profundidad o altura del árbol de análisis sintáctico de $\varphi$.
  \item[$b$)] Números de conectivos de una fórmula: \textit{con}$(\varphi)$
    devuelve el número de conectivos de $\varphi$.
  \item[$c$)] 
  \item[$d$)] 
  \item[$e$)] 
  \item[$f$)] 
  \item[$g$)] 
  \end{itemize}
\end{enumerate}
\end{document}
