\documentclass{article}

% Symbols
\usepackage{amsfonts, amsthm}
\usepackage{upgreek}
\usepackage{physics}
\usepackage{cancel}
\usepackage{amssymb, latexsym, amsmath}

%Algorithms
\usepackage[ruled,lined,linesnumbered,commentsnumbered]{algorithm2e}

%% Identación
\setlength{\parindent}{0cm}

% Código
\newcommand{\code}[1]{\textcolor{white!25!black}{\texttt{#1}}}
\usepackage{listings}

%AMS
\usepackage{amsthm}
\newtheorem{algo-thm}{Algoritmo}

% Proof
\renewcommand*{\proofname}{\textbf{Demostraci\'on:}}
% Theorem
\newtheorem*{theorem}{Teorema}

% Graphics
\usepackage{graphicx}
\usepackage{pgf}

% Color a letras.
%\usepackage[usenames,dvipsnames,svgnames,table]{xcolor}

% << >>
\usepackage[T1]{fontenc}

% Tikz
\usepackage{tkz-graph}
\usepackage{tikz}
\usetikzlibrary{arrows,automata}
\usepackage{tikz}
\usetikzlibrary{arrows,automata}
%\usetikzlibrary[topaths]

% Def. Dr. César.
\usetikzlibrary{shapes,calc}
\tikzstyle{edge}=[shorten <=2pt, shorten >=2pt, >=stealth, line width=1.1pt]
\tikzstyle{blueE}=[shorten <=2pt, shorten >=2pt, >=stealth, line width=1.5pt, blue]
\tikzstyle{blackV}=[circle, fill=black, minimum size=6pt, inner sep=0pt, outer sep=0pt]
\tikzstyle{blueV}=[circle, fill=blue, draw, minimum size=6pt, line width=0.75pt, inner sep=0pt, outer sep=0pt]
\tikzstyle{redV}=[circle, fill=red, draw, minimum size=6pt, line width=0.75pt, inner sep=0pt, outer sep=0pt]
\tikzstyle{redSV}=[semicircle, fill=red, minimum size=3pt, inner sep=0pt, outer sep=0pt, rotate=225]
\tikzstyle{blueSV}=[semicircle, fill=blue, minimum size=3pt, inner sep=0pt, outer sep=0pt, rotate=225]
\tikzstyle{blackSV}=[semicircle, fill=black, minimum size=3pt, inner sep=0pt, outer sep=0pt, rotate=225]
\tikzstyle{vertex}=[circle, draw, minimum size=6pt, line width=0.75pt, inner sep=0pt, outer sep=0pt]

% Margins
\addtolength{\voffset}{-1.5cm}
\addtolength{\hoffset}{-1.5cm}
\addtolength{\textwidth}{3cm}
\addtolength{\textheight}{3cm}

%Header-Footer
\usepackage{fancyhdr}
\renewcommand{\headrulewidth}{1pt}

\newcommand{\set}[1]{
  \left\{ #1 \right\}
}

%\pagenumbering{gobble} -- Este comando
%                       -- quita el número de página.
\footskip = 50pt
\renewcommand{\headrulewidth}{1pt}

\pagestyle{fancyplain}

\begin{document}
\title{UNIVERSIDAD AUT\'ONOMA DE M\'EXICO\\ Facultad de Ciencias}
\author{Autor: Adri\'an Aguilera Moreno}
\date{}
\maketitle
\begin{center}
  \includegraphics[scale=0.20]{../Imagen/Portada.jpg}\\[0.4cm]
  \Large
  \bf{Lógica Computacional}
  \normalsize
\end{center}
\newpage
\fancyhead[r]{ Lógica Computacional 2022-2}
%%%%%%%%%%%%%%%%%%%%%%%%%%%%%%%%%%%%%%%%%%%%%%%%%%%%%
\section*{\LARGE{Semanal 2}}
\begin{enumerate}
  %%%%%%%%%%%%%%%%%%%%%%%%%%%%%%%% Ejercicio 01  %%%%%%%%%%%%%%%%%%%%%%%%%%%%%%%%
\item Para cada uno de los siguientes ejercicios, \textbf{justifica ampliamente}
  tu respuesta.
  \begin{itemize}
  \item[$a$)] \textbf{Define recursivamente} la función \textit{atom}$(\phi)$ que,
    para $\phi \in PL$, regrese el número de fórmulas atómicas ($\top, \bot,$ o variables
    proposicionales) en $\phi$.
  \item[$b$)] \textbf{Define recursivamente} la función \textit{con}$(\phi)$ que,
    para $\phi \in PL$, regresa el número de conectivos lógicos en $\phi$.
  \item[$c$)] \textbf{Demuestra} que para cualquier fórmula $\phi \in PL$ se cumple que
    \[
    atom(\phi) \leq con(\phi) + 1
    \]
    Debes usar las funciones que definiste en los dos incisos anteriores.
  \end{itemize}
  %%%%%%%%%%%%%%%%%%%%%%%%%%%%%%%% Ejercicio 02  %%%%%%%%%%%%%%%%%%%%%%%%%%%%%%%%
\item Para cada uno de los siguientes ejercicios, \textbf{justifica ampliamente}
  tu respuesta.
  \begin{itemize}
  \item[$a$)] \textbf{Define recursivamente} la función \textit{icd}$(\phi)$ que,
    para $\phi \in PL$, regresa la fórmula resultante de intercambiar en $\phi$
    todas las conjunciones por disyunciones y todas las disyunciones por conjunciones.
  \item[$b$)] \textbf{Verifica} la definición de tu función mostrando paso a paso
    la ejecución de
    \[
    icd(p \land (q \lor \neg r) \rightarrow \neg (r \lor s) \land t)
    \]
  \end{itemize}
\end{enumerate}

  %%%%%%%%%%%%%%%%%%%%%%%%%%%%%%%% Extra  %%%%%%%%%%%%%%%%%%%%%%%%%%%%%%%%
\textbf{Desafío extra...}

En una granja con mucho folklore se discute acerca del siguiente razonamiento.

\hspace*{0.5cm} \guillemotleft El día que nace un becerro, cualquiera lo puede cargar con
facilidad. Y los becerros no crecen demasiado en un día, entonces si puedes cargar a un
becerro un día, lo puedes cargar también al día siguiente. Siguiendo con este razonamiento,
entonces también debería serte posible cargar al becerro el día siguiente y el siguiente y
así sucesivamente. Pero después de un año, el becerro se va a convertir en una vaca adulta
de 1000kg, algo que claramente ya no puedes cargar.\guillemotright

\hspace*{0.5cm} \textbf{Muestra}, si es posible, que el argumento es correcto utilizándo
inducción. En caso contrario, \textbf{indica} en donde está el error en el razonamiento
inductivo. \textbf{Justifica ampliamente tu respuesta}.
\end{document}
