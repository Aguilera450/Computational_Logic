\documentclass{article}

% Symbols
\usepackage{amsfonts, amsthm}
\usepackage{upgreek}
\usepackage{physics}
\usepackage{cancel}
\usepackage{amssymb, latexsym, amsmath}

%Algorithms
\usepackage[ruled,lined,linesnumbered,commentsnumbered]{algorithm2e}

%% Identación
\setlength{\parindent}{0cm}

% Código
\newcommand{\code}[1]{\textcolor{white!25!black}{\texttt{#1}}}
\usepackage{listings}

%AMS
\usepackage{amsthm}
\newtheorem{algo-thm}{Algoritmo}

% Proof
\renewcommand*{\proofname}{\textbf{Demostraci\'on:}}
% Theorem
\newtheorem*{theorem}{Teorema}

% Graphics
\usepackage{graphicx}
\usepackage{pgf}

% Color a letras.
%\usepackage[usenames,dvipsnames,svgnames,table]{xcolor}

% << >>
\usepackage[T1]{fontenc}

% Tikz
\usepackage{tkz-graph}
\usepackage{tikz}
\usetikzlibrary{arrows,automata}
\usepackage{tikz}
\usetikzlibrary{arrows,automata}
%\usetikzlibrary[topaths]

% Def. Dr. César.
\usetikzlibrary{shapes,calc}
\tikzstyle{edge}=[shorten <=2pt, shorten >=2pt, >=stealth, line width=1.1pt]
\tikzstyle{blueE}=[shorten <=2pt, shorten >=2pt, >=stealth, line width=1.5pt, blue]
\tikzstyle{blackV}=[circle, fill=black, minimum size=6pt, inner sep=0pt, outer sep=0pt]
\tikzstyle{blueV}=[circle, fill=blue, draw, minimum size=6pt, line width=0.75pt, inner sep=0pt, outer sep=0pt]
\tikzstyle{redV}=[circle, fill=red, draw, minimum size=6pt, line width=0.75pt, inner sep=0pt, outer sep=0pt]
\tikzstyle{redSV}=[semicircle, fill=red, minimum size=3pt, inner sep=0pt, outer sep=0pt, rotate=225]
\tikzstyle{blueSV}=[semicircle, fill=blue, minimum size=3pt, inner sep=0pt, outer sep=0pt, rotate=225]
\tikzstyle{blackSV}=[semicircle, fill=black, minimum size=3pt, inner sep=0pt, outer sep=0pt, rotate=225]
\tikzstyle{vertex}=[circle, draw, minimum size=6pt, line width=0.75pt, inner sep=0pt, outer sep=0pt]

% Margins
\addtolength{\voffset}{-1.5cm}
\addtolength{\hoffset}{-1.5cm}
\addtolength{\textwidth}{3cm}
\addtolength{\textheight}{3cm}

%Header-Footer
\usepackage{fancyhdr}
\renewcommand{\headrulewidth}{1pt}

\newcommand{\set}[1]{
  \left\{ #1 \right\}
}

%\pagenumbering{gobble} -- Este comando
%                       -- quita el número de página.
\footskip = 50pt
\renewcommand{\headrulewidth}{1pt}

\pagestyle{fancyplain}

\begin{document}
\title{UNIVERSIDAD AUT\'ONOMA DE M\'EXICO\\ Facultad de Ciencias}
\author{Autor: Adri\'an Aguilera Moreno}
\date{}
\maketitle
\begin{center}
  \includegraphics[scale=0.20]{../Imagen/Portada.jpg}\\[0.4cm]
  \Large
  \bf{Lógica Computacional}
  \normalsize
\end{center}
\newpage
\fancyhead[r]{ Lógica Computacional 2022-2}
%%%%%%%%%%%%%%%%%%%%%%%%%%%%%%%%%%%%%%%%%%%%%%%%%%%%%
\section*{\LARGE{Semanal 2}}
\begin{enumerate}
  %%%%%%%%%%%%%%%%%%%%%%%%%%%%%%%% Ejercicio 01  %%%%%%%%%%%%%%%%%%%%%%%%%%%%%%%%
\item Para cada uno de los siguientes ejercicios, \textbf{justifica ampliamente}
  tu respuesta.
  \begin{itemize}
  \item[$a$)] \textbf{Define recursivamente} la función \textit{atom}$(\phi)$ que,
    para $\phi \in PL$, regrese el número de fórmulas atómicas ($\top, \bot,$ o variables
    proposicionales) en $\phi$.
  \item[$b$)] \textbf{Define recursivamente} la función \textit{con}$(\phi)$ que,
    para $\phi \in PL$, regresa el número de conectivos lógicos en $\phi$.
  \item[$c$)] \textbf{Demuestra} que para cualquier fórmula $\phi \in PL$ se cumple que
    \[
    atom(\phi) \leq con(\phi) + 1
    \]
    Debes usar las funciones que definiste en los dos incisos anteriores.
  \end{itemize}
  
  $\triangledown$ \textbf{Solución:}
  \begin{itemize}
  \item[$a$)] Sea $\varphi \in ATOM$, definimos $atom(\varphi) = 1$. De igual manera,
    tenemos que
    \[
    atom(\top) = 1 = atom(\bot)
    \]
    Para $\phi, \gamma$ fórmulas en $PL$, definimos para los casos no atómicos (casos con
    llamada recursiva), esto es
    \begin{eqnarray*}
      atom(\neg \phi) &=& atom(\phi)\\
      atom(\phi \land \gamma) &=& atom(\phi) + atom(\gamma)\\
      atom(\phi \lor \gamma) &=& atom(\phi) + atom(\gamma)\\
      atom(\phi \Rightarrow \gamma) &=& atom(\phi) + atom(\gamma)\\
      atom(\phi \Leftrightarrow \gamma) &=& atom(\phi) + atom(\gamma)
    \end{eqnarray*}
    Así, $atom: PL \rightarrow \mathbb{N}/\{0\}$. Nótese que el codominio de ``$atom$'' no puede
    contener al $0$, pues $\{\{\neg\}, \{\land\}, \{\lor\}, \{\Rightarrow\}, \{\Leftrightarrow\}\}
    \nsubseteq PL$\footnote{Por la definición recursiva de $PL$, el caso base se da cuando $\varphi
    \in ATOM$.}.
  \item[$b$)] Para $\varphi \in ATOM \cup \{\bot, \top\}$, definimos $con(\varphi) = 0$. Luego, para
    $\phi, \gamma \in PL$, tenemos que
    \begin{eqnarray*}
      con(\neg \phi) &=& 1 + con(\phi)\\
      con(\phi \land \gamma) &=& 1 + con(\phi) + con(\gamma)\\
      con(\phi \lor \gamma) &=& 1 + con(\phi) + con(\gamma)\\
      con(\phi \Rightarrow \gamma) &=& 1 + con(\phi) + con(\gamma)\\
      con(\phi \Leftrightarrow \gamma) &=& 1 + con(\phi) + con(\gamma)
    \end{eqnarray*}
    De lo anterior podemos constatar que, $con: PL \rightarrow \mathbb{N}$.
    
    \begin{center}
      \fbox{
        \begin{minipage}[b][1\height]%
          [t]{0.867\textwidth}
          Nótese que, como
          \[
          \{\{\neg\}, \{\land\}, \{\lor\}, \{\Rightarrow\}, \{\Leftrightarrow\}\} \nsubseteq PL
          \]
          entonces no tendremos cosas como $con(\neg) = 1 = con(\land) = con(\lor) =
          con(\Rightarrow) = con(\Leftrightarrow)$, es por esto que no entran en nuestro
          caso base de la recursión.        
      \end{minipage}}
    \end{center}
  \item[$c$)] \begin{proof}
    Sea $\varphi \in ATOM \cup \{\bot, \top\}$, así
    \begin{eqnarray*}
      &\Rightarrow& atom(\varphi) \leq  con(\varphi) + 1
      \hspace*{1cm} \text{Es lo que queremos ver.}\\
      &\Rightarrow& atom(\varphi) \leq 0 + 1
      \hspace*{1.9cm} \text{Por definición de ``\textit{con}''.}\\
      &\Rightarrow& 1 \leq 1
      \hspace*{3.5cm} \text{Por definición de ``\textit{atom}''.}
    \end{eqnarray*}
    Por la dicotomía de ``$\leq$'', tenemos que $1 = 1$ y se cumple lo anterior.
    
    Ahora, supongamos que para $\phi, \gamma$ fórmulas en $PL$ se cumple que
    \begin{eqnarray*}
      atom(\phi) &\leq& con(\phi) + 1\\
      atom(\gamma) &\leq& con(\gamma) + 1
    \end{eqnarray*}
    luego, observemos los siguientes casos:
    \begin{itemize}
    \item[$\cdot$)] Negación. Supongamos sin pérdida de generalidad a $\phi$, así
      \begin{eqnarray*}
        atom(\neg \phi) &=& atom(\phi)
        \hspace*{2.1cm} \text{Definición de ``\textit{atom}''.}\\
        &\leq& con(\phi) + 1
        \hspace*{1.3cm} \text{Por hipótesis de inducción.}\\
        &=& con(\neg \phi)
        \hspace*{2cm} \text{Por definición de ``\textit{con}''.}\\
        &\leq& con(\neg \phi) + 1
        \hspace*{0.5cm} \text{Por dicotomía de ``$\leq$'', tenemos ``$<$''.}
      \end{eqnarray*}
      \[
      \therefore \; atom(\neg \phi) \leq con(\phi) + 1
      \]
    \item[$\cdot$)] Conjunción. Veamos que
      \begin{eqnarray*}
        atom(\phi \land \gamma) &=& atom(\phi) + atom(\gamma)
        \hspace*{2.1cm} \text{Por definición de ``$atom$''.}\\
        &\leq& (con(\phi) + 1) + (con(\gamma) + 1)
        \hspace*{1cm} \text{Por hipótesis inductiva.}\\
        &=& (1 + con(\phi) + con(\gamma)) + 1
        \hspace*{1.1cm} \text{Asociatividad en la suma.}\\
        &=& con(\phi \land \gamma) + 1
        \hspace*{3.4cm} \text{Definición de ``\textit{con}''.}
      \end{eqnarray*}
      \[
      \therefore \; atom(\phi \land \gamma) \leq con(\phi \land \gamma) + 1
      \]
    \item[$\cdot$)] Disyunción. Observemos que
      \begin{eqnarray*}
        atom(\phi \lor \gamma) &=& atom(\phi) + atom(\gamma)
        \hspace*{2.1cm} \text{Por definición de ``$atom$''.}\\
        &\leq& (con(\phi) + 1) + (con(\gamma) + 1)
        \hspace*{1cm} \text{Por hipótesis inductiva.}\\
        &=& (1 + con(\phi) + con(\gamma)) + 1
        \hspace*{1.1cm} \text{Asociatividad en la suma.}\\
        &=& con(\phi \lor \gamma) + 1
        \hspace*{3.4cm} \text{Definición de ``\textit{con}''.}
      \end{eqnarray*}
      \[
      \therefore \; atom(\phi \lor \gamma) \leq con(\phi \lor \gamma) + 1
      \]
    \item[$\cdot$)] Implicación simple. Tenemos que
      \begin{eqnarray*}
        atom(\phi \Rightarrow \gamma) &=& atom(\phi) + atom(\gamma)
        \hspace*{2.1cm} \text{Por definición de ``$atom$''.}\\
        &\leq& (con(\phi) + 1) + (con(\gamma) + 1)
        \hspace*{1cm} \text{Por hipótesis inductiva.}\\
        &=& (1 + con(\phi) + con(\gamma)) + 1
        \hspace*{1.1cm} \text{Asociatividad en la suma.}\\
        &=& con(\phi \Rightarrow \gamma) + 1
        \hspace*{3.3cm} \text{Definición de ``\textit{con}''.}
      \end{eqnarray*}
      \[
      \therefore \; atom(\phi \Rightarrow \gamma) \leq con(\phi \Rightarrow \gamma) + 1
      \]
    \item[$\cdot$)] Bicondicional. Notemos que
      \begin{eqnarray*}
        atom(\phi \Leftrightarrow \gamma) &=& atom(\phi) + atom(\gamma)
        \hspace*{2.1cm} \text{Por definición de ``$atom$''.}\\
        &\leq& (con(\phi) + 1) + (con(\gamma) + 1)
        \hspace*{1cm} \text{Por hipótesis inductiva.}\\
        &=& (1 + con(\phi) + con(\gamma)) + 1
        \hspace*{1.1cm} \text{Asociatividad en la suma.}\\
        &=& con(\phi \Leftrightarrow \gamma) + 1
        \hspace*{3.3cm} \text{Definición de ``\textit{con}''.}
      \end{eqnarray*}
      \[
      \therefore \; atom(\phi \Leftrightarrow \gamma) \leq con(\phi \Leftrightarrow \gamma) + 1
      \]
    \end{itemize}
  \end{proof}
  \end{itemize}
%
\hfill $\lhd$
%%%%%%%%%%%%%%%%%%%%%%%%%%%%%%%% Ejercicio 02  %%%%%%%%%%%%%%%%%%%%%%%%%%%%%%%%
\item Para cada uno de los siguientes ejercicios, \textbf{justifica ampliamente}
  tu respuesta.
  \begin{itemize}
  \item[$a$)] \textbf{Define recursivamente} la función \textit{icd}$(\phi)$ que,
    para $\phi \in PL$, regresa la fórmula resultante de intercambiar en $\phi$
    todas las conjunciones por disyunciones y todas las disyunciones por conjunciones.
  \item[$b$)] \textbf{Verifica} la definición de tu función mostrando paso a paso
    la ejecución de
    \[
    icd(p \land (q \lor \neg r) \rightarrow \neg (r \lor s) \land t)
    \]
  \end{itemize}
  $\triangledown$ \textbf{Solución:}
  Analicemos ambos incisos:
  \begin{itemize}
  \item[$\square$] Sea $\varphi \in ATOM$, entonces definimos a $icd(\varphi) = \varphi$. En
    general, para $\phi, \gamma$ fórmulas en $PL$ se tiene que
    \begin{eqnarray*}
      icd(\neg \phi) &=& \neg icd(\phi)\\
      icd(\phi \land \gamma) &=& icd(\phi) \lor icd(\gamma)\\
      icd(\phi \lor \gamma) &=& icd(\phi) \land icd(\gamma)\\
      icd(\phi \Rightarrow \gamma) &=& icd(\phi) \Rightarrow icd(\gamma)\\
      icd(\phi \Leftrightarrow \gamma) &=& icd(\phi) \Leftrightarrow icd(\gamma)
    \end{eqnarray*}
    así, $icd: PL \rightarrow PL$ [por la definición recursiva de $PL$].
  \item[$\square$] Ahora, verifiquemos la definición anterior con el siguiente
    caso particular, esto es
  \end{itemize}
  \begin{eqnarray*}
    icd(p \land (q \lor \neg r) \rightarrow \neg (r \lor s) \land t) &=&
    icd(p \land (q \lor \neg r)) \rightarrow icd(\neg (r \lor s) \land t))\\
    &=& icd(p) \lor icd((q \lor \neg r)) \rightarrow icd(\neg (r \lor s)) \lor icd(t)\\
    &=& p \lor (icd(q) \land icd(\neg r)) \rightarrow \neg icd((r \lor s)) \lor t\\
    &=& p \lor (q \land \neg icd(r)) \rightarrow \neg (icd(r) \land icd(s)) \lor t\\
    &=& p \lor (q \land \neg r) \rightarrow \neg (r \land s) \lor t
  \end{eqnarray*}
  \hfill $\lhd$
\end{enumerate}

  %%%%%%%%%%%%%%%%%%%%%%%%%%%%%%%% Extra  %%%%%%%%%%%%%%%%%%%%%%%%%%%%%%%%
\textbf{Desafío extra...}

En una granja con mucho folklore se discute acerca del siguiente razonamiento.

\hspace*{0.5cm} \guillemotleft El día que nace un becerro, cualquiera lo puede cargar con
facilidad. Y los becerros no crecen demasiado en un día, entonces si puedes cargar a un
becerro un día, lo puedes cargar también al día siguiente. Siguiendo con este razonamiento,
entonces también debería serte posible cargar al becerro el día siguiente y el siguiente y
así sucesivamente. Pero después de un año, el becerro se va a convertir en una vaca adulta
de 1000kg, algo que claramente ya no puedes cargar.\guillemotright

\hspace*{0.5cm} \textbf{Muestra}, si es posible, que el argumento es correcto utilizándo
inducción. En caso contrario, \textbf{indica} en donde está el error en el razonamiento
inductivo. \textbf{Justifica ampliamente tu respuesta}.
\\

$\triangledown$ \textbf{Solución:}
Observemos que el enunciado tiene como conclusión al predicado: ``Pero después de un año,
el becerro se va a convertir en una vaca adulta de 1000kg, algo que claramente ya no puedes
cargar'', que se resume en ``El becerro se va a convertir en una vaca adulta de 1000kg''
\textbf{y} ``algo que claramente ya no puedes cargar'', si nosotros suponemos que el argumento
en general es válido, entonces se debe cumplir la conjunción anterior, entonces concluimos que
\underline{no se puede cargar al becerro (vaca) después de un año} [en la conjunción ambos
  conjuntos deben tener un valor de verdad $1$ o verdadero para que el enunciado sea válido].


Así mostremos que el becerro después de un año aún se podrá cargar, esto por inducción en la
cantidad de días que pasan.

\textbf{Caso base:} ``El día que nace un becerro, cualquiera lo puede cargar con facilidad''.

\textbf{Hipótesis de Inducción:} ``debería serte posible cargar al becerro el día siguiente
y el siguiente y así sucesivamente''.

\textbf{Paso de inducción:} Supongamos un día $x$ donde el becerro se pueda cargar, entonces
el $sig(x) = y$ es un día, y por hipótesis de inducción tenemos que en el día $y$ se puede
cargar el becerro. Así, todos los días se puede cargar al becerro, en particular cuando este
tenga más de un año!!, he aquí un contradicción, pues habíamos notado que después de un año
la vaca ya no podría ser cargada.
\\

Pero ¿dónde surge el error? El error surge de la hipótesis de inducción, si vemos al argumento
como algo definido recursivamente, entonces podemos notar que su definición está hecha sin mayor
cuidado, pues se hace en base a los días transcurrido sin percatarse del peso que adquiere día
con día, y si bien la hipótesis es cierta para los primeros días, se deja de cumplir cuando
el becerro alcanza un peso mayor a lo que es humanamente posible de cargar. Así, definir que el
becerro se puede cargar el siguiente de un día en donde se puede cargar es incorrecto.
\hfill $\lhd$
\end{document}
