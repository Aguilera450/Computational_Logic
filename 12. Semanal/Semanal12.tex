\documentclass{article}

% Symbols
\usepackage{amsfonts, amsthm}
\usepackage{upgreek}
\usepackage{physics}
\usepackage{cancel}
\usepackage{amssymb, latexsym, amsmath}
\usepackage{amsmath}
\usepackage{listings}

%Algorithms
\usepackage[ruled,lined,linesnumbered,commentsnumbered]{algorithm2e}

%% Semantic
\usepackage[inference]{semantic}

%% Identación
\setlength{\parindent}{0cm}

% Código
\newcommand{\code}[1]{\textcolor{white!25!black}{\texttt{#1}}}
\usepackage{listings}

%AMS
\usepackage{amsthm}
\newtheorem{algo-thm}{Algoritmo}

% Proof
\renewcommand*{\proofname}{\textbf{Demostraci\'on:}}

% Theorem
\newtheorem*{theorem}{Teorema}

% Hipervínculos:
\usepackage{hyperref}

% Graphics
\usepackage{graphicx}
\usepackage{pgf}

% Color a letras.
%\usepackage[usenames,dvipsnames,svgnames,table]{xcolor}

% Tikz
\usepackage{tkz-graph}
\usepackage{tikz}
\usetikzlibrary{arrows,automata}
\usepackage{tikz}
\usetikzlibrary{arrows,automata}
%\usetikzlibrary[topaths]

% Def. Dr. César.
\usetikzlibrary{shapes,calc}
\tikzstyle{edge}=[shorten <=2pt, shorten >=2pt, >=stealth, line width=1.1pt]
\tikzstyle{blueE}=[shorten <=2pt, shorten >=2pt, >=stealth, line width=1.5pt, blue]
\tikzstyle{blackV}=[circle, fill=black, minimum size=6pt, inner sep=0pt, outer sep=0pt]
\tikzstyle{blueV}=[circle, fill=blue, draw, minimum size=6pt, line width=0.75pt, inner sep=0pt, outer sep=0pt]
\tikzstyle{redV}=[circle, fill=red, draw, minimum size=6pt, line width=0.75pt, inner sep=0pt, outer sep=0pt]
\tikzstyle{redSV}=[semicircle, fill=red, minimum size=3pt, inner sep=0pt, outer sep=0pt, rotate=225]
\tikzstyle{blueSV}=[semicircle, fill=blue, minimum size=3pt, inner sep=0pt, outer sep=0pt, rotate=225]
\tikzstyle{blackSV}=[semicircle, fill=black, minimum size=3pt, inner sep=0pt, outer sep=0pt, rotate=225]
\tikzstyle{vertex}=[circle, draw, minimum size=6pt, line width=0.75pt, inner sep=0pt, outer sep=0pt]

% Margins
\addtolength{\voffset}{-1.5cm}
\addtolength{\hoffset}{-1.5cm}
\addtolength{\textwidth}{3cm}
\addtolength{\textheight}{3cm}

%Header-Footer
\usepackage{fancyhdr}
\renewcommand{\headrulewidth}{1pt}

\newcommand{\set}[1]{
  \left\{ #1 \right\}
}

%\pagenumbering{gobble} -- Este comando
%                       -- quita el número de página.
\footskip = 50pt
\renewcommand{\headrulewidth}{1pt}

\pagestyle{fancyplain}

%% Bibliografía APA
\usepackage[backend=biber]{biblatex}
\bibliography{./Bibliografia/BaseDatos}

\begin{document}
\title{UNIVERSIDAD NACIONAL AUT\'ONOMA DE M\'EXICO\\ Facultad de Ciencias}
\author{Autor: Adri\'an Aguilera Moreno}

\date{}
\maketitle
\begin{center}
  \includegraphics[scale=0.20]{../Imagen/Portada.jpg}\\[0.4cm]
  \Large
  \bf{Lógica Computacional}
  \normalsize
\end{center}
\newpage
\fancyhead[r]{ Lógica Computacional 2022-2}
%%%%%%%%%%%%%%%%%%%%%%%%%%%%%%%%%%%%%%%%%%%%%%%%%%%%%
\section*{\LARGE{Semanal 12}}
Para cada uno de los siguientes ejercicios, \textbf{justifica ampliamente} tu respuesta.
\newline
\newline
\textbf{Desmuestra} la consecuencia lógica de los siguientes usando tácticas:
\newcommand{\localtextbulletone}{\textcolor{black}{\raisebox{.45ex}{\rule{.6ex}{.6ex}}}}
\renewcommand{\labelitemi}{\localtextbulletone}
\begin{itemize}
\item
  \[\mathrm{H:\; (A \rightarrow C)\land(B \rightarrow C) \vdash A \lor B \rightarrow C}\]
\item
  \[\mathrm{H_1\;:\; \exists_{x} Qx,\; H_2\;:\; \forall_{x}(Qx \land \exists_{y} Py
    \rightarrow Qfx),\; H_3\;:\;\forall_{z}(Qz \rightarrow Qgz) \vdash Pb \rightarrow \exists_{w} Qfgw}\]
\end{itemize}
Demostración para el primer punto (diagrama de árbol)\footnote{Se realiza razonamiento hacia atrás.}:
\[
\inference*[(Intro)]{
  \inference*[(destruct)]{
    \inference*[(Apply)]{\inference*[]{\text{Assumption} \checkmark}{\Gamma, A \vdash A }
     }{\hspace*{0.5cm} \Gamma, A \vdash C \hspace*{0.5cm}}
    &
    \inference*[(Apply)]{
    \inference*[]{\text{Assumption} \checkmark}{\Gamma, B \vdash B}}{\hspace*{.5cm} \Gamma, B \vdash C \hspace*{.5cm}}}
             {\hspace*{3cm} \Gamma, A \lor B \vdash C \hspace*{3cm}}}
           {\hspace*{1.2cm}\Gamma \vdash A \lor B \rightarrow C\hspace*{0.5cm}}
           \]
Para $\Gamma = \{A \rightarrow C, B \rightarrow C\}$. 

De manera secuencial sería
\begin{enumerate}
\item \code{Intro}: $\vdash (A \rightarrow C) \rightarrow (B \rightarrow C) \rightarrow A \lor B \rightarrow C
  \;\; \rhd\;\; (A \rightarrow C) \vdash (B \rightarrow C) \rightarrow A \lor B \rightarrow C$
\item \code{Intro}: $(A \rightarrow C) \vdash (B \rightarrow C) \rightarrow A \lor B \rightarrow C
  \hspace*{0.65cm}\rhd\;\; (A \rightarrow C), (B \rightarrow C) \vdash A \lor B \rightarrow C$
\item \code{Intro}: $(A \rightarrow C), (B \rightarrow C) \vdash A \lor B \rightarrow C
  \hspace*{1.05cm}\rhd\;\; (A \rightarrow C), (B \rightarrow C), (A \lor B) \vdash C$
\item \code{destruc}: $(A \rightarrow C), (B \rightarrow C), (A \lor B) \vdash C
  \hspace*{0.825cm}\rhd\;\; \Gamma, A \vdash C \; ;\; \Gamma, B \vdash C$
\item \code{Apply}: $\;\Gamma, A \vdash C \; ;\; \Gamma, B \vdash C \hspace*{3cm}\rhd\;\; \Gamma, A \vdash A \; ;\; \Gamma, B \vdash B$
\item \code{Assumption}: $\;\Gamma, A \vdash A \; ;\; \Gamma, B \vdash B \hspace*{2.1cm}\rhd\; \square$
\end{enumerate}
Algunas observaciones:
\newline
1. En la demostración por árbol se dan por hecho los pasos 1 y 2.
\newline
2. El paso 4 destruye una disyunción.
\newline
3. El paso 5 es realmente un \code{Apply} simultáneo.
\newline
4. El paso 6 es relamente un \code{Assumption} simultáneo.
\newline
5. Me faltó especificar a que hipótesis se le aplica cada regla, pero en el árbol las marco.
\hfill $\square$
\newline
\newline
Demostración para el punto 2 (diagrama de árbol):
\[
\inference*[]{
  \inference*[]{
    \inference*[]{
      \inference*[]{
        \inference*[]{
          \inference*[]{
            \inference*[]{
              \inference*[]{\text{Assumption \checkmark}}{\Gamma'', Qt \vdash Qt}
            }{\Gamma'' \vdash Qt \;\; \text{(Destruct)}}
          }{\Gamma', Qt \rightarrow Qgt \vdash Qgt \;\; \text{(Apply)}}
        }{\hspace*{0.8cm} \Gamma' \vdash Qgt \;\; \text{(Destruct)} \hspace*{0.5cm}}
        &
        \inference*[]{
          \inference*[]{\text{Assumption} \checkmark}{\Gamma' \vdash Pb}
        }{\Gamma' \vdash \exists_y Py \;\; \text{(Exists)}}
      }{\hspace*{2cm} \Gamma, Qgt \land \exists_y Py \rightarrow Qfgt, Pb \vdash Qfgt\;\; \text{(Apply)} \hspace*{2cm}}
    }{\hspace*{3.5cm} \Gamma, Pb \vdash Qfgt \;\; \text{(Destruct)}\hspace*{3.5cm}}
  }{\Gamma, Pb \vdash \exists_{w}Qfgw \;\; \text{(Exists)}}
}{\Gamma \vdash Pb \rightarrow \exists_{w}Qfgw \;\;\text{(Intro)}}
\]
Para $\Gamma = \{\exists_x Qx, \forall_x (Qx \land \exists_y Py \rightarrow Qfx), \forall_z(Qz \rightarrow Qgz)\}$,
$\Gamma' = \{\Gamma, Qgt \land \exists_y Py \rightarrow Qfgt, Pb\}$, y $\Gamma'' = \{\Gamma', Qt \rightarrow Qgt\}$.
\newline
Así, de manera secuencial sería
\begin{enumerate}
\item \code{Intro}: $\Gamma \vdash Pb \rightarrow \exists_{w}Qfgw
  \hspace*{3.3cm} \rhd \hspace*{0.3cm}
  \Gamma, Pb \vdash \exists_{w}Qfgw$
\item \code{Exists}: $\Gamma, Pb \vdash \exists_{w}Qfgw
  \hspace*{3.55cm} \rhd \hspace*{0.3cm}
  \Gamma, Pb \vdash Qfgt$
\item \code{Destruct}: $\Gamma, Pb \vdash Qfgt
  \hspace*{3.75cm} \rhd \hspace*{0.3cm}
  \Gamma, Qgt \land \exists_y Py \rightarrow Qfgt, Pb \vdash Qfgt$
\item \code{Apply}: $\Gamma, Qgt \land \exists_y Py \rightarrow Qfgt, Pb \vdash Qfgt
  \hspace*{1cm} \rhd \hspace*{0.3cm}
  \Gamma' \vdash Qgt\;\; ;\;\; \Gamma' \vdash \exists_y Py$
\item \code{Exists}: $\Gamma' \vdash Qgt\;\; ;\;\; \Gamma' \vdash \exists_y Py
  \hspace*{2.72cm} \rhd \hspace*{0.3cm}
  \Gamma' \vdash Qgt\;\; ;\;\; \Gamma' \vdash Pb$
\item \code{Assumption}: $\Gamma' \vdash Qgt\;\; ;\;\; \Gamma' \vdash Pb
  \hspace*{2.4cm} \rhd \hspace*{0.3cm}
  \Gamma' \vdash Qgt\;\; ;\;\; \square$
\item \code{Destruct}: $\Gamma' \vdash Qgt
  \hspace*{4.5cm} \rhd \hspace*{0.3cm}
  \Gamma', Qt \rightarrow Qgt \vdash Qgt$
\item \code{Apply}: $\Gamma', Qt \rightarrow Qgt \vdash Qgt
  \hspace*{3.35cm} \rhd \hspace*{0.3cm}
  \Gamma'' \vdash Qt$
\item \code{Destruct}: $\Gamma'' \vdash Qt
  \hspace*{4.6cm} \rhd \hspace*{0.3cm}
  \Gamma'', Qt \vdash Qt$
\item \code{Assumption}: $\Gamma'', Qt \vdash Qt
  \hspace*{3.67cm} \rhd \hspace*{0.3cm} \square$
\end{enumerate}
Algunas observaciones:
\newline
1. En el paso 2 se tiene un \code{Exists} con $t$.
\newline
2. El paso 3 destruye el para todo $H_2$.
\newline
3. En el paso 4 se aplica la hipótesis obtenida del paso 2.
Aquí\footnote{En el paso 4.} se hacen dos pasos a la vez, pues se aplica el destructor
de la conjunción.
\newline
4. El paso 6 funciona porque $Pb$ es parte de $\Gamma'$. Así, solo nos queda concluir
por una de las ramas.
\newline
5. En el paso 7 se destruye el para todo de $H_3$.
\newline
6. El paso 8 aplica la nueva hipótesis obtenida del paso 7.
\newline
7. En el paso 9 se destruye el existencial de $H_1$. \hfill $\square$
\newline
\newline
\textbf{Nota}: los demás pasos son más claros, no especifiqué más por el formato en el
que dibujé los diagramas y realice los pasos en secuencia. Si llegara haber alguna duda
agradecería que me preguntarán, pero entendería sino lo hicieran, en el examen trataré
de hacer mir árboles más explícitos para no tener que explicar después.
\end{document}
