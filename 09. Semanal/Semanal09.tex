\documentclass{article}

% Symbols
\usepackage[T1]{fontenc}
\usepackage{upgreek}
\usepackage{physics}
\usepackage{cancel}
\usepackage{amsfonts, amsthm}
\usepackage{amssymb, latexsym, amsmath}


% Proof
\renewcommand*{\proofname}{\textbf{Soluci\'on:}}

% Theorem
\newtheorem*{theorem}{Teorema}

%Algorithms
\usepackage[ruled,lined,linesnumbered,commentsnumbered]{algorithm2e}

%% Identación
\setlength{\parindent}{0cm}

% Código
\newcommand{\code}[1]{\textcolor{white!25!black}{\texttt{#1}}}
\usepackage{listings}

%AMS
\usepackage{amsthm}
\newtheorem{algo-thm}{Algoritmo}

% Graphics
\usepackage{graphicx}
\usepackage{pgf}

% Margins
\addtolength{\voffset}{-1.5cm}
\addtolength{\hoffset}{-1.5cm}
\addtolength{\textwidth}{3cm}
\addtolength{\textheight}{3cm}

%Header-Footer
\usepackage{fancyhdr}
\renewcommand{\headrulewidth}{1pt}

\newcommand{\set}[1]{
  \left\{ #1 \right\}
}

\footskip = 50pt
\renewcommand{\headrulewidth}{1pt}

\pagestyle{fancyplain}

\begin{document}
\title{UNIVERSIDAD NACIONAL AUT\'ONOMA DE M\'EXICO\\ Facultad de Ciencias}
\author{Autor: Adri\'an Aguilera Moreno}
\date{}
\maketitle
\begin{center}
  \includegraphics[scale=0.20]{../Imagen/Portada.jpg}\\[0.4cm]
  \Large
  \bf{Lógica Computacional}
  \normalsize
\end{center}
\newpage
\fancyhead[r]{ Lógica Computacional 2022-2}
%%%%%%%%%%%%%%%%%%%%%%%%%%%%%%%%%%%%%%%%%%%%%%%%%%%%%
\section*{\LARGE{Semanal 9}}
Para cada uno de los siguientes ejercicios, \textbf{justifica ampliamente} tu respuesta.

\vspace*{0.5cm}
1. \textbf{Decide} si los siguientes conjuntos son unificables, aplicando paso a paso el algoritmo de
Martelli-Montanari.
\newcommand{\localtextbulletone}{\textcolor{black}{\raisebox{.45ex}{\rule{.6ex}{.6ex}}}}
\renewcommand{\labelitemi}{\localtextbulletone}
\begin{itemize}
  % -------------------------------- Ejercicio 1.1
\item $W = \{Q\big(a,x, g(a,b,c)\big), Q\big(a, f(x), g(a,b,y)), Q\big(a, f(f(w)\big), g(a, x, g(c, b, a)\big)\}$
  \begin{proof} Ejecutando el algoritmo de \textbf{Martelli-Montanari}, tenemos que
    
    \textbf{I}.
    \begin{eqnarray}
      & \{Q\big(a, x, g(a, b, c)\big) = Q\big(a, f(f(w)), g(a, x, g(c, b, a))\big)\} &\\
      & \{a = a, x = f(f(w)), g(a,b,c) = g(a, x, g(c, b, a))\} &\\
      & \{x = f(f(w)), g(a,b,c) = g(a, x, g(c, b, a))\} &\\
      & \{f(f(w)) = f(f(w)), g(a,b,c) = g(a, x, g(c, b, a))\} &\\
      & \{f(w) = f(w), g(a,b,c) = g(a, f(f(w)), g(c, b, a))\} &\\
      & \{w = w, g(a,b,c) = g(a, f(f(w)), g(c, b, a))\} &\\
      & \{g(a,b,c) = g(a, f(f(w)), g(c, b, a))\} &\\
      & \{a = a, b = f(f(w)), c = g(c, b, a)\} &\\
      & \{b = f(f(w)), c = g(c, b, a)\} &\\
      & \{f(f(w)) = b, c = g(c, b, a)\} &\\
      & \square &
    \end{eqnarray}
    2. Descomposición.

    3. Eliminación.

    4. Sustitución $[x := f(f(w))]$.

    5. Descomposición.

    6. Descomposición.

    7. Eliminación.

    8. Descomposición.

    9. Eliminación.
    
    10. Swap.
    
    11. Descomposición fállida.

    \[\therefore\;\; W \text{ no es unificable.}\]
  \end{proof}
  \newpage
  % -------------------------------- Ejercicio 1.2
\item $W = \{P\big(x, f(y)\big), P\big(g(y,a), f(b)\big), P\big(g(b,z), w\big)\}$
  \begin{proof} Ejecutando el algoritmo de \textbf{Martelli-Montanari}, tenemos que
    
    \textbf{I}.
    \begin{eqnarray*}
      & \{P\big(g(y,a), f(b)\big) = P\big(g(b,z), w\big)\} &\\
      & \{g(y,a) = g(b,z), f(b) = w\} &
      \hspace*{1cm} \text{Descomposición.}\\
      & \{y = b, a = z, f(b) = w \} &
      \hspace*{1cm} \text{Descomposición.}\\
      & \{b = b, a = z, f(b) = w\} &
      \hspace*{0.8cm} \text{Sustitución } [y := b].\\
      & \{a = z, f(b) = w\} &
      \hspace*{1.4cm} \text{Eliminación.}\\
      & \{z = a, f(b) = w\} &
      \hspace*{1.9cm} \text{Swap.}\\
      & \{a = a, f(b) = w\} &
      \hspace*{0.75cm} \text{Sustitución } [z := a].\\
      & \{f(b) = w\} &
      \hspace*{1.5cm} \text{Eliminación.}\\
      & \{f(b) = f(b)\} &
      \hspace*{0.55cm} \text{Sustitución } [w := f(b)].\\
      & \{b = b\} &
      \hspace*{1.18cm} \text{Descomposición.}\\
      & \emptyset &
      \hspace*{1.5cm} \text{Eliminación.}
    \end{eqnarray*}
    \[ \therefore\;\; \mu_{1} = [y := b][z := a][w := f(b)].\]
    \textbf{II}. Aplicamos la sustitución a $W$\footnote{Nótese que al ser un conjunto, este no
    admite elementos repetidos.} la sustitución $\mu_{1}$. Así,
    \[\Rightarrow W \mu_{1} = W' = \{P(x, f(b)), P(g(b, a), f(b))\}.\]
    \textbf{III}.
    \begin{eqnarray*}
      & \{P\big(x, f(b)\big) = P\big(g(b,a), f(b)\big)\} &\\
      & \{x = g(b,a), f(b) = f(b)\} &
      \hspace*{1.72cm} \text{Descomposición.}\\
      & \{g(b,a) = g(b,a), f(b) = f(b)\} &
      \hspace*{1cm} \text{Sustitución } [x := g(b,a)].\\
      & \{b = b, a = a, f(b) = f(b)\} &
      \hspace*{1.71cm} \text{Descomposición.}\\
      & \{a = a, f(b) = f(b)\} &
      \hspace*{2.1cm} \text{Eliminación.}\\
      & \{f(b) = f(b)\} &
      \hspace*{2.1cm} \text{Eliminación.}\\
      & \{b = b\} &
      \hspace*{1.8cm} \text{Descomposición.}\\
      & \emptyset &
      \hspace*{2.1cm} \text{Eliminación.}
    \end{eqnarray*}
    \[\therefore\;\; \text{W es unificable.}\]
    \[\therefore\;\; \mu = \mu_{1}[x := g(b,a)] \text{ es el \textbf{UMG} (la composición de las sustituciones) de } W.\]
  \end{proof}
\end{itemize}
% -------------------------------------- Ejercicio 2
\newpage
2. Usando el algoritmo de unificación de Martelli-Montanari, encuentra para el siguiente
conjunto de expresiones el \textbf{umg}, y si no tiene, explica por qué. Indica la regla que se
utiliza en cada paso del algoritmo.

\begin{eqnarray*}
  E = \{S\big(f(z), x, g(h(a, b)), g(c)\big), S\big(f(g(y)), z, w, g(y)\big), S\big(u, v, w, z\big)\}
\end{eqnarray*}
\begin{proof} Ejecutando el algoritmo de \textbf{Martelli-Montanari}, tenemos que

  \textbf{I}.
  \begin{eqnarray*}
    & \{S\big(f(z), x, g(h(a,b)), g(c)\big) = S\big(u, v, w, z\big)\} &\\
    & \{f(z) = u, x = v, g(h(a,b)) = w, g(c) = z\} &
    \hspace*{1cm} \text{Descomposición.}\\
    & \{u = f(z), x = v, g(h(a,b)) = w, g(c) = z\} &
    \hspace*{1.8cm} \text{Swap.}\\
    & \{f(z) = f(z), x = v, g(h(a,b)) = w, g(c) = z\} &
    \hspace*{0.6cm} \text{Sustitución } [u := f(z)].\\
    & \{z = z, x = v, g(h(a,b)) = w, g(c) = z\} &
    \hspace*{1.1cm} \text{Descomposición.}\\
    & \{x = v, g(h(a,b)) = w, g(c) = z\} &
    \hspace*{1.45cm} \text{Eliminación.}\\
    & \{v = v, g(h(a,b)) = w, g(c) = z\} &
    \hspace*{0.8cm} \text{Sustitución } [x := v].\\
    & \{g(h(a,b)) = w, g(c) = z\} &
    \hspace*{1.45cm} \text{Eliminación.}\\
    & \{w = g(h(a,b)), g(c) = z\} &
    \hspace*{1.95cm} \text{Swap.}\\
    & \{g(h(a,b)) = g(h(a,b)), g(c) = z\} &
    \hspace*{0.2cm} \text{Sustitución } [w := g(h(a, b))].\\
    & \{h(a,b) = h(a,b), g(c) = z\} &
    \hspace*{1.2cm} \text{Descomposición.}\\
    & \{a = a, b = b, g(c) = z\} &
    \hspace*{1.2cm} \text{Descomposición.}\\
    & \{b = b, g(c) = z\} &
    \hspace*{1.45cm} \text{Eliminación.}\\
    & \{g(c) = z\} &
    \hspace*{1.45cm} \text{Eliminación.}\\
    & \{z = g(c)\} &
    \hspace*{1.94cm} \text{Swap.}\\
    & \{g(c) = g(c)\} &
    \hspace*{0.7cm} \text{Sustitución } [z := g(c)].\\
    & \{c = c\} &
    \hspace*{1.2cm} \text{Descomposición.}\\
    & \emptyset &
    \hspace*{1.45cm} \text{Eliminación.}
  \end{eqnarray*}
  \begin{eqnarray*}
    \therefore\;\; \sigma_{1} &=& [u := f(z)][x := v][w := g(h(a, b))][z := g(c)]\\
    &=& [u, x, w, z := f(g(c)), v, g(h(a, b)), g(c)].
  \end{eqnarray*}
  \textbf{II}. Aplicamos la sustitución a $E$ la sustitución $\sigma_{1}$ . Así,
  \begin{eqnarray*}
    \Rightarrow E \sigma_{1} &=& E'\\
    &=& \{S\big(f(g(c)), v, g(h(a, b)), g(c)\big) = S\big(f(g(y), g(c), g(h(a,b))), g(y)\big)\}
  \end{eqnarray*}
  \textbf{III}.
  \setcounter{equation}{0}
  \begin{eqnarray}
    & \{S\big(f(g(c)), v, g(h(a, b)), g(c)\big) = S\big(f(g(y)), g(c), g(h(a,b))), g(y)\big)\} &\\
    & \{f(g(c)) = f(g(y)), v = g(c), g(h(a, b)) = g(h(a,b))), g(c) = g(y)\} &\\
    & \{g(c) = g(y), v = g(c), g(h(a, b)) = g(h(a,b))), g(c) = g(y)\} &\\
    & \{c = y, v = g(c), g(h(a, b)) = g(h(a,b))), g(c) = g(y)\} &\\
    & \{y = c, v = g(c), g(h(a, b)) = g(h(a,b))), g(c) = g(y)\} &\\
    & \{c = c, v = g(c), g(h(a, b)) = g(h(a,b))), g(c) = g(y)\} &\\
    & \{v = g(c), g(h(a, b)) = g(h(a,b))), g(c) = g(y)\} &\\
    & \{g(c) = g(c), g(h(a, b)) = g(h(a,b))), g(c) = g(y)\} &\\
    & \{c = c, g(h(a, b)) = g(h(a,b))), g(c) = g(c)\} &\\
    & \{g(h(a, b)) = g(h(a,b))), g(c) = g(c)\} &\\
    & \{h(a, b) = h(a,b)), g(c) = g(c)\} &\\
    & \{a = a, b = b, g(c) = g(c)\} &\\
    & \{b = b, g(c) = g(c)\} &\\
    & \{ g(c) = g(c)\} &\\
    & \{ c = c\} &\\
    & \emptyset &
  \end{eqnarray}
  2. Descomposición.
  
  3. Descomposición. 
  
  4. Descomposición.
  
  5. Swap.
  
  6. Sustitución $[y := c]$.
  
  7. Eliminación.
  
  8. Sustitución $[v := g(c)]$.
  
  9. Descomposición.
  
  10. Eliminación.
  
  11. Descomposición.
  
  12. Descomposición.
  
  13. Eliminación.
  
  14. Eliminación.
  
  15. Descomposición.
  
  16. Eliminación.
  \[\therefore\;\; W \text{ es unificable.}\]
  \begin{eqnarray*}
    \sigma &=& \sigma_{1}[y := c][v := g(c)]\\
    &=& [u, x, w, z := f(g(c)), v, g(h(a, b)), g(c)][y := c][v := g(c)]\\
    &=& [u, x, w, z, y, v := f(g(c)), g(c), g(h(a, b)), g(c), c, g(c)].
  \end{eqnarray*}
  Se obvia que $\sigma$ es el UMG. Pues, cualquier sustitución extra funge como
  identidad y se cumple que para alguna sustitución $\tau$ sucede que $\sigma\tau = \sigma$.
  
  Además, el algoritmo de unificación de Martelli-Montanari nos garantiza que regresa el unificador
  más general.
\end{proof}
\end{document}
