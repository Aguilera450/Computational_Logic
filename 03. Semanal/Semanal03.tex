\documentclass{article}

% Symbols
\usepackage{amsfonts, amsthm}
\usepackage{upgreek}
\usepackage{physics}
\usepackage{cancel}
\usepackage{amssymb, latexsym, amsmath}

%Algorithms
\usepackage[ruled,lined,linesnumbered,commentsnumbered]{algorithm2e}

%% Identación
\setlength{\parindent}{0cm}

% Código
\newcommand{\code}[1]{\textcolor{white!25!black}{\texttt{#1}}}
\usepackage{listings}

%AMS
\usepackage{amsthm}
\newtheorem{algo-thm}{Algoritmo}

% Proof
\renewcommand*{\proofname}{\textbf{Demostraci\'on:}}
% Theorem
\newtheorem*{theorem}{Teorema}

% Graphics
\usepackage{graphicx}
\usepackage{pgf}

% Color a letras.
%\usepackage[usenames,dvipsnames,svgnames,table]{xcolor}

% << >>
\usepackage[T1]{fontenc}

% Tikz
\usepackage{tkz-graph}
\usepackage{tikz}
\usetikzlibrary{arrows,automata}
\usepackage{tikz}
\usetikzlibrary{arrows,automata}
%\usetikzlibrary[topaths]

% Def. Dr. César.
\usetikzlibrary{shapes,calc}
\tikzstyle{edge}=[shorten <=2pt, shorten >=2pt, >=stealth, line width=1.1pt]
\tikzstyle{blueE}=[shorten <=2pt, shorten >=2pt, >=stealth, line width=1.5pt, blue]
\tikzstyle{blackV}=[circle, fill=black, minimum size=6pt, inner sep=0pt, outer sep=0pt]
\tikzstyle{blueV}=[circle, fill=blue, draw, minimum size=6pt, line width=0.75pt, inner sep=0pt, outer sep=0pt]
\tikzstyle{redV}=[circle, fill=red, draw, minimum size=6pt, line width=0.75pt, inner sep=0pt, outer sep=0pt]
\tikzstyle{redSV}=[semicircle, fill=red, minimum size=3pt, inner sep=0pt, outer sep=0pt, rotate=225]
\tikzstyle{blueSV}=[semicircle, fill=blue, minimum size=3pt, inner sep=0pt, outer sep=0pt, rotate=225]
\tikzstyle{blackSV}=[semicircle, fill=black, minimum size=3pt, inner sep=0pt, outer sep=0pt, rotate=225]
\tikzstyle{vertex}=[circle, draw, minimum size=6pt, line width=0.75pt, inner sep=0pt, outer sep=0pt]

% Margins
\addtolength{\voffset}{-1.5cm}
\addtolength{\hoffset}{-1.5cm}
\addtolength{\textwidth}{3cm}
\addtolength{\textheight}{3cm}

%Header-Footer
\usepackage{fancyhdr}
\renewcommand{\headrulewidth}{1pt}

\newcommand{\set}[1]{
  \left\{ #1 \right\}
}

%\pagenumbering{gobble} -- Este comando
%                       -- quita el número de página.
\footskip = 50pt
\renewcommand{\headrulewidth}{1pt}

\pagestyle{fancyplain}

\begin{document}
\title{UNIVERSIDAD AUT\'ONOMA DE M\'EXICO\\ Facultad de Ciencias}
\author{Autor: Adri\'an Aguilera Moreno}
\date{}
\maketitle
\begin{center}
  \includegraphics[scale=0.20]{../Imagen/Portada.jpg}\\[0.4cm]
  \Large
  \bf{Lógica Computacional}
  \normalsize
\end{center}
\newpage
\fancyhead[r]{ Lógica Computacional 2022-2}
%%%%%%%%%%%%%%%%%%%%%%%%%%%%%%%%%%%%%%%%%%%%%%%%%%%%%
\section*{\LARGE{Semanal 3}}
Para cada uno de los siguientes ejercicios, \textbf{justifica ampliamente} tu respuesta.
\begin{enumerate}
\item \textbf{Realiza} las siguientes sustituciones \textbf{eliminando} los paréntesis
  superfluos en el resultado y \textbf{mostrando paso a paso} el procedimiento.
  \begin{itemize}
  \item[$a$)] $((q \lor r)[q, p := \neg p, s] \rightarrow (r \land \neg(r \leftrightarrow p)))[p, r, q := r \lor q, q \land p, s]$.
  \item[$b$)] $(u \lor t) \rightarrow (\neg r \leftrightarrow (u \leftrightarrow s))[r, u, t := u, t, r]$.
  \end{itemize}
\item
  \begin{itemize}
  \item[$a$)] \textbf{Define recursivamente} la función $pa$ que dada una fórmula $\varphi$, devuelve el número
    de paréntesis abiertos ``$($'' que tiene $\varphi$.
  \item[$b$)] \textbf{Define recursivamente} la función $pc$ que dada una fórmula $\varphi$, devuelve el número
    de paréntesis cerrados ``$)$'' que tiene $\varphi$.
  \item[$c$)] Sea $\varphi = (((\neg p \land q) \lor \neg r) \rightarrow r)$. Demuestra que
    \[
    pa(\varphi) - pc(\varphi) = 0
    \]
  \end{itemize}
\end{enumerate}
\textbf{Desafío extra...}
\renewcommand{\labelitemi}{$-$}
\begin{itemize}
\item \textbf{Define recursivamente} una función compress que elimina los elementos consecutivos
  repetidos de una lista.

  Ejemplo:
  \begin{center}
    \code{compress([a,a,a,a,i,i,i,u,d,d,d,d,d,a,a,a,a,a,a,a]) = [a,i,u,d,a]}
  \end{center}
\item \textbf{Demuestra}, usando tu definición, que:
  \begin{center}
    \code{compress([1, 2, 2, 3, 3, 3]) = [1, 2, 3]}
  \end{center}
\end{itemize}
\end{document}
