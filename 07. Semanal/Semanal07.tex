\documentclass{article}

% Symbols
\usepackage[T1]{fontenc}
\usepackage{upgreek}
\usepackage{physics}
\usepackage{cancel}
\usepackage{amsfonts, amsthm}
\usepackage{amssymb, latexsym, amsmath}


% Proof
\renewcommand*{\proofname}{\textbf{Soluci\'on:}}

% Theorem
\newtheorem*{theorem}{Teorema}

%Algorithms
\usepackage[ruled,lined,linesnumbered,commentsnumbered]{algorithm2e}

%% Identación
\setlength{\parindent}{0cm}

% Código
\newcommand{\code}[1]{\textcolor{white!25!black}{\texttt{#1}}}
\usepackage{listings}

%AMS
\usepackage{amsthm}
\newtheorem{algo-thm}{Algoritmo}

% Graphics
\usepackage{graphicx}
\usepackage{pgf}

% Margins
\addtolength{\voffset}{-1.5cm}
\addtolength{\hoffset}{-1.5cm}
\addtolength{\textwidth}{3cm}
\addtolength{\textheight}{3cm}

%Header-Footer
\usepackage{fancyhdr}
\renewcommand{\headrulewidth}{1pt}

\newcommand{\set}[1]{
  \left\{ #1 \right\}
}

\footskip = 50pt
\renewcommand{\headrulewidth}{1pt}

\pagestyle{fancyplain}

\begin{document}
\title{UNIVERSIDAD NACIONAL AUT\'ONOMA DE M\'EXICO\\ Facultad de Ciencias}
\author{Autor: Adri\'an Aguilera Moreno}
\date{}
\maketitle
\begin{center}
  \includegraphics[scale=0.20]{../Imagen/Portada.jpg}\\[0.4cm]
  \Large
  \bf{Lógica Computacional}
  \normalsize
\end{center}
\newpage
\fancyhead[r]{ Lógica Computacional 2022-2}
%%%%%%%%%%%%%%%%%%%%%%%%%%%%%%%%%%%%%%%%%%%%%%%%%%%%%
\section*{\LARGE{Semanal 7}}
Para cada uno de los ejercicios, \textbf{justifica ampliamente} tu respuesta.

\begin{enumerate}
  %%%%%%%%%%%%%%%%%%%%%%%%%%%%%%%%%%%%%%%%%%%%%%%%%%%%% Ejercicio 01.
\item Para cada una de las siguientes fórmulas, \textbf{encuentra} una
  fórmula que sea $ \alpha$-equivalente tal que cada cuantificador ligue a una
  variable distinta.
  \newcommand{\localtextbulletone}{\textcolor{black}{\raisebox{.45ex}{\rule{.6ex}{.6ex}}}}
  \renewcommand{\labelitemi}{\localtextbulletone}
  \begin{itemize}
  \item $\varphi = \forall_{v}. (\exists_{y}.(S(u,y,z))) \lor \exists_{z}.(P(f(v), g(y), z))$
  \item $\phi = \exists_{x}.(\forall_{y}.(R(x,y) \rightarrow Q(z)) \land Q(x))$
  \item $\gamma = P(x,y) \land \exists_{y}.(Q(x,y) \leftrightarrow P(y,y))$
  \item $\psi = \forall_{x}.(P(y,z) \rightarrow \exists_{y}.(R(x,y,w) \lor S(a,y)))$
    \begin{proof}
      \begin{eqnarray*}
        \varphi &\sim_{\alpha}& \forall_{v}. (\exists_{r}.(S(u,r,z))) \lor \exists_{z}.(P(f(v), g(y), z))\\
        &\sim_{\alpha}& \forall_{v}. (\exists_{r}.(S(u,r,z))) \lor \exists_{p}.(P(f(v), g(y), p)).\\
        \phi &\sim_{\alpha}& \exists_{v}. (\forall_{y}. (R(v,y)) \rightarrow Q(z)) \land Q(v)\\
        &\sim_{\alpha}& \exists_{v}. (\forall_{u}. (R(v,u)) \rightarrow Q(z)) \land Q(v).\\
        \gamma &\sim_{\alpha}& P(x,y) \land \exists_{m}.(Q(x, m) \leftrightarrow P(m,m)).\\
        \psi &\sim_{\alpha}& \forall_{v}.(P(y,z) \rightarrow \exists_{y}.(R(v, y, w) \lor S(a,y)))\\
        &\sim_{\alpha}& \forall_{v}.(P(y,z) \rightarrow \exists_{u}.(R(v, u, w) \lor S(a,u))).
      \end{eqnarray*}
    \end{proof}
  \end{itemize}
  %%%%%%%%%%%%%%%%%%%%%%%%%%%%%%%%%%%%%%%%%%%%%%%%%%%%% Ejercicio 02.
\item \textbf{Realiza} las siguientes sustituciones:
  \begin{itemize}
  \item $\varphi [w := f(u), u := g(y, z), y := b]$
  \item $\phi[x := g(h(a,y)), y := g(b), z := f(y)]$
  \item $\gamma[x,y,z := y, f(g(a)), f(b)]$
  \item $\psi[x,y,w := y,f(c),x]$
  \end{itemize}
  \begin{proof}
    Sustitución en $\varphi$. Sean $\beta = [w := f(u)]$, $\eta = [u := g(y,z)]$ y
    $\theta = [y := b]$. Entonces,
    \begin{eqnarray*}
      [\forall_{v}. (\exists_{r}.(S(u,r,z))) \lor \exists_{p}.(P(f(v), g(y), p))] \beta \eta \theta
      &=& [\forall_{v}. (\exists_{r}.(S(u,r,z))) \lor \exists_{p}.(P(f(v), g(y), p))]\eta \theta\\
      &=& [\forall_{v}. (\exists_{r}.(S(g(y,z),r,z))) \lor \exists_{p}.(P(f(v), g(y), p))] \theta\\
      &=& \forall_{v}. (\exists_{r}.(S(g(b,z),r,z))) \lor \exists_{p}.(P(f(v), g(b), p)).
    \end{eqnarray*}
    Notemos que se omite la aplicación del algoritmo de sustitución, pues en la tarea semanal $6$
    se ejemplifica paso a paso este proceso.
    
    Sustitución en $\phi$. Sean $\beta = [x := g(h(a,y))]$, $\eta = [y := g(b)]$ y
    $\theta = [z := f(y)]$. Entonces,
    \begin{eqnarray*}
      [\exists_{v}. (\forall_{u}. (R(v,u)) \rightarrow Q(z)) \land Q(v)]\beta \eta \theta &=&
      [\exists_{v}. (\forall_{u}. (R(v,u)) \rightarrow Q(z)) \land Q(v)]\eta \theta \\
      &=& [\exists_{v}. (\forall_{u}. (R(v,u)) \rightarrow Q(z)) \land Q(v)] \theta\\
      &=& \exists_{v}. (\forall_{u}. (R(v,u)) \rightarrow Q(f(y))) \land Q(v).
    \end{eqnarray*}
    Mismo caso que el anterior.
    
    Sustitución en $\gamma$. Sea $\eta = [x,y,z := y, f(g(a)), f(b)]$. Entonces,
    \begin{eqnarray*}
      [P(x,y) \land \exists_{m}.(Q(x, m) \leftrightarrow P(m,m))] \eta &=&
      \underbrace{P(x,y)}_{\eta} \land \underbrace{\exists_{m}.(Q(x, m) \leftrightarrow P(m,m))}_{\eta}\\
      &=& P(\underbrace{x}_{\eta},\underbrace{y}_{\eta}) \land \exists_{m}.(\underbrace{Q(x, m)}_{\eta} \leftrightarrow \underbrace{P(m,m)}_{\eta})\\
      &=&
      P(y,f(g(a))) \land \exists_{m}.(Q(\underbrace{x}_{\eta}, \underbrace{m}_{\eta}) \leftrightarrow P(\underbrace{m}_{\eta},\underbrace{m}_{\eta}))\\
      &=&  P(y,f(g(a))) \land \exists_{m}.(Q(y, m) \leftrightarrow P(m,m)).
    \end{eqnarray*}
    
    Sustitución en $\psi$. Sea $\eta = [x,y,w := y, f(c), x]$. Entonces,
    \begin{eqnarray*}
      [\forall_{v}.(P(y,z) \rightarrow \exists_{u}.(R(v, u, w) \lor S(a,u)))] \eta &=&
       \forall_{v}.(\underbrace{P(y,z) \rightarrow \exists_{u}.(R(v, u, w) \lor S(a,u))}_{\eta})\\
       &=& \forall_{v}.(\underbrace{P(y,z)}_{\eta} \rightarrow \underbrace{\exists_{u}.(R(v, u, w) \lor S(a,u))}_{\eta})\\
       &=& \forall_{v}.(P(\underbrace{y}_{\eta}, \underbrace{z}_{\eta}) \rightarrow \exists_{u}.(\underbrace{R(v, u, w)}_{\eta} \lor \underbrace{S(a,u)}_{\eta}))\\
       &=& \forall_{v}.(P(f(c), z) \rightarrow \exists_{u}.(R(\underbrace{v}_{\eta}, \underbrace{u}_{\eta}, \underbrace{w}_{\eta}) \lor S(\underbrace{a}_{\eta},\underbrace{u}_{\eta})))\\
       &=& \forall_{v}.(P(f(c), z) \rightarrow \exists_{u}.(R(v, u, x) \lor S(a,u))).
    \end{eqnarray*}
  \end{proof}
\end{enumerate}
\end{document}

