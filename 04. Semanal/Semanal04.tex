\documentclass{article}

% Symbols
\usepackage{amsfonts, amsthm}
\usepackage{upgreek}
\usepackage{physics}
\usepackage{cancel}
\usepackage{amssymb, latexsym, amsmath}

%Algorithms
\usepackage[ruled,lined,linesnumbered,commentsnumbered]{algorithm2e}

%% Identación
\setlength{\parindent}{0cm}

% Código
\newcommand{\code}[1]{\textcolor{white!25!black}{\texttt{#1}}}
\usepackage{listings}

%AMS
\usepackage{amsthm}
\newtheorem{algo-thm}{Algoritmo}

% Proof
\renewcommand*{\proofname}{\textbf{Soluci\'on:}}
% Theorem
\newtheorem*{theorem}{Teorema}

% Graphics
\usepackage{graphicx}
\usepackage{pgf}

% Color a letras.
%\usepackage[usenames,dvipsnames,svgnames,table]{xcolor}

% << >>
\usepackage[T1]{fontenc}

% Tikz
\usepackage{tkz-graph}
\usepackage{tikz}
\usetikzlibrary{arrows,automata}
\usepackage{tikz}
\usetikzlibrary{arrows,automata}
%\usetikzlibrary[topaths]

% Def. Dr. César.
\usetikzlibrary{shapes,calc}
\tikzstyle{edge}=[shorten <=2pt, shorten >=2pt, >=stealth, line width=1.1pt]
\tikzstyle{blueE}=[shorten <=2pt, shorten >=2pt, >=stealth, line width=1.5pt, blue]
\tikzstyle{blackV}=[circle, fill=black, minimum size=6pt, inner sep=0pt, outer sep=0pt]
\tikzstyle{blueV}=[circle, fill=blue, draw, minimum size=6pt, line width=0.75pt, inner sep=0pt, outer sep=0pt]
\tikzstyle{redV}=[circle, fill=red, draw, minimum size=6pt, line width=0.75pt, inner sep=0pt, outer sep=0pt]
\tikzstyle{redSV}=[semicircle, fill=red, minimum size=3pt, inner sep=0pt, outer sep=0pt, rotate=225]
\tikzstyle{blueSV}=[semicircle, fill=blue, minimum size=3pt, inner sep=0pt, outer sep=0pt, rotate=225]
\tikzstyle{blackSV}=[semicircle, fill=black, minimum size=3pt, inner sep=0pt, outer sep=0pt, rotate=225]
\tikzstyle{vertex}=[circle, draw, minimum size=6pt, line width=0.75pt, inner sep=0pt, outer sep=0pt]

% Margins
\addtolength{\voffset}{-1.5cm}
\addtolength{\hoffset}{-1.5cm}
\addtolength{\textwidth}{3cm}
\addtolength{\textheight}{3cm}

%Header-Footer
\usepackage{fancyhdr}
\renewcommand{\headrulewidth}{1pt}

\newcommand{\set}[1]{
  \left\{ #1 \right\}
}

%\pagenumbering{gobble} -- Este comando
%                       -- quita el número de página.
\footskip = 50pt

\pagestyle{fancyplain}

\begin{document}
\title{UNIVERSIDAD NACIONAL AUT\'ONOMA DE M\'EXICO\\ Facultad de Ciencias}
\author{Autor: Adri\'an Aguilera Moreno}
\date{}
\maketitle
\begin{center}
  \includegraphics[scale=0.20]{../Imagen/Portada.jpg}\\[0.4cm]
  \Large
  \bf{Lógica Computacional}
  \normalsize
\end{center}
\newpage
\fancyhead[r]{ Lógica Computacional 2022-2}
%%%%%%%%%%%%%%%%%%%%%%%%%%%%%%%%%%%%%%%%%%%%%%%%%%%%%
\section*{\LARGE{Semanal 4}}
Para cada uno de los ejercicios, \textbf{justifica ampliamente} tu respuesta.
\begin{itemize}
  %%%%%%%%%%%%%%%%%%%%%%%%%%%%%%%%%%%%%%%%%%%%%%%%%%%%% Ejercicio 01.  
\item[1.] \textbf{Obten} la Forma Normal Negativa de las siguientes fórmulas:
  \begin{itemize}
  \item[$\cdot$)] $\varphi = ((p \rightarrow r) \land (q \rightarrow r)) \rightarrow ((p \land q) \rightarrow r).$
  \item[$\cdot$)] $\phi = (p \land (q \rightarrow r)) \rightarrow s.$
  \end{itemize}
  
  \begin{proof} Analicemos dos posibles casos:
    
    1. Solucionemos $\varphi = ((p \rightarrow r) \land (q \rightarrow r)) \rightarrow ((p \land q) \rightarrow r)$,
    veamos que
    \begin{eqnarray*}
      p \rightarrow r &\equiv& \neg p \lor r.
      \hspace*{2cm} \text{Por eliminación: } \varphi \rightarrow \psi \equiv \neg \varphi \lor \psi.\\
      q \rightarrow r &\equiv& \neg q \lor r.
      \hspace*{2cm} \text{Por eliminación: } \varphi \rightarrow \psi \equiv \neg \varphi \lor \psi.\\
      (p \land q) \rightarrow r &\equiv& \neg (p \land q) \lor r
      \hspace*{1.25cm} \text{Por eliminación: } \varphi \rightarrow \psi \equiv \neg \varphi \lor \psi.\\
      &\equiv& \neg p \lor \neg q \lor r
      \hspace*{1.27cm} \text{Por De Morgan: } \neg (\varphi \land \psi) \equiv \neg \varphi \lor \neg \psi.\\
    \end{eqnarray*}
    de lo anterior se sigue que
    \[
    ((p \rightarrow r) \land (q \rightarrow r)) \rightarrow ((p \land q) \rightarrow r)
    \equiv (\neg p \lor r) \land (\neg q \lor r) \rightarrow \neg p \lor \neg q \lor r
    \]
    empleando nuevamente eliminación [$\varphi \rightarrow \psi \equiv \neg \varphi \lor \psi$]
    tenemos que
    \begin{eqnarray*}
    (\neg p \lor r) \land (\neg q \lor r) \rightarrow \neg p \lor \neg q \lor r &\equiv_{1}&
      \neg ((\neg p \lor r) \land (\neg q \lor r)) \lor \neg p \lor \neg q \lor r\\
      &\equiv_{2}& \neg (\neg p \lor r) \lor \neg (\neg q \lor r) \lor \neg p \lor \neg q \lor r\\
      &\equiv_{3}& (\neg\neg p \land \neg r) \lor (\neg\neg q \land \neg r) \lor \neg p \lor \neg q \lor r\\
      &\equiv_{4}& (p \land \neg r) \lor (q \land \neg r) \lor \neg p \lor \neg q \lor r\\
      &\equiv_{5}& (\neg r \land (p \lor q)) \lor \neg p \lor \neg q \lor r
    \end{eqnarray*}
    $1^{\text{ra}}$ línea por eliminación de la implicación.
    
    $2^{\text{da}}$ y $3^{\text{ra}}$ línea por De Morgan.
    
    $4^{\text{ta}}$ línea por doble negación.
    
    $5^{\text{ta}}$ línea por distributividad.
    
    \[
    \therefore\;\;  \varphi \equiv (\neg r \land (p \lor q)) \lor \neg p \lor \neg q \lor r
    \]
    2. Llevemos a $\phi = (p \land (q \rightarrow r)) \rightarrow s$ a FNN, veamos que
    \begin{eqnarray*}
      \phi &\equiv& \neg (p \land (q \rightarrow r)) \lor s
      \hspace*{2cm} \text{Por eliminación: } \varphi \rightarrow \psi \equiv \neg \varphi \lor \psi.\\
      &\equiv& (\neg p \lor \neg (q \rightarrow r)) \lor s
      \hspace*{2cm} \text{Distributividad de la negación.}\\
      &\equiv& (\neg p \lor (q \land \neg r)) \lor s
      \hspace{2.5cm} \text{Negación de la implicación.}
    \end{eqnarray*}
    \[
    \therefore\;\;  \phi \equiv (\neg p \lor (q \land \neg r)) \lor s
    \]
  \end{proof}
  \newpage
\item[2.] \textbf{Obten} la Forma Normal Conjuntiva de las siguientes fórmulas:
  \begin{itemize}
  \item[$\cdot$)] $\varphi = ((p \rightarrow r) \rightarrow q) \land (r \rightarrow q).$
  \item[$\cdot$)] $\phi = \neg p \land q \rightarrow p \land (r \rightarrow q).$
  \end{itemize}
  \begin{proof} Ahora, analicemos dos posibles casos:
    
    1. Pasando a la FNC la fórmula $\varphi = ((p \rightarrow r) \rightarrow q) \land (r \rightarrow q)$,
    tenemos
    \begin{eqnarray*}
      \varphi &\equiv& (\neg (q \rightarrow r)  \lor q) \land (\neg r \lor q)
      \hspace*{0.9cm} \text{Por eliminación: } \varphi \rightarrow \psi \equiv \neg \varphi \lor \psi.\\
      &\equiv& ((q \land \neg r)  \lor q) \land (\neg r \lor q)
      \hspace*{1.8cm} \text{Negación de implicación.}\\
      &\equiv& (q \lor q) \land (q \lor \neg r) \land (\neg r \lor q)
      \hspace{2cm} \text{Distributividad.}\\
      &\equiv& q \land (q \lor \neg r)
      \hspace{4.7cm} \text{Idenpotencia.}\\
      &\equiv& q
      \hspace{6.5cm} \text{Absorción.}\\
    \end{eqnarray*}
    \[\therefore\;\;  \varphi \equiv q\]
    Obs. En clase con la profesora se discutió acerca de FNC y FND, en ambos
    casos, se concluyó que se aceptan $\bot, \top,$ y algún $x \in VarProp$
    como casos triviales de estas formas normales. En adelante se omite esta
    observación.
      
    2. Pasando a FNC por medio de equivalencias lógicas a la función
    $\phi = \neg p \land q \rightarrow p \land (r \rightarrow q)$,
    tenemos que
    \begin{eqnarray*}
      \phi &\equiv& \neg (\neg p \land q) \lor (p \land (\neg r \lor q))
      \hspace*{2cm} \text{Eliminación de la implicación.}\\
      &\equiv& (p \lor \neg q) \lor (p \land \neg r) \lor (p \land q)
      \hspace*{1cm} \text{Leyes de De Morgan y distributividad.}\\
      &\equiv& p \lor (p \land \neg r) \lor \neg q \lor (p \land q)
      \hspace*{2.5cm} \text{Conmutatividad de } \lor.\\
      &\equiv& ((p \lor p) \land (p \lor \neg r)) \lor ((\neg q \lor q) \land (p \lor \neg q))
      \hspace*{0.7cm} \text{Distributividad.}\\
      &\equiv& (p \land (p \lor \neg r)) \lor (\top \land (p \lor \neg q))
      \hspace*{1.5cm} \text{Idempotencia y $3^{\text{ro}}$ excluido.}\\
      &\equiv&  p \lor (p \lor \neg q)
      \hspace*{4.2cm} \text{Neutro conjuntivo y absorción.}\\
      &\equiv& p
      \hspace*{7.5cm} \text{Absorción.}
    \end{eqnarray*}
    \[\therefore\;\;  \phi \equiv p\]
  \end{proof}
\newpage
\item[3.] \textbf{Obten} la Forma Normal Disyuntiva de las siguientes fórmulas:
  \begin{itemize}
  \item[$\cdot$)] $\neg(w \rightarrow \neg p) \lor \neg ((\neg s \leftrightarrow w) \lor (p \land s)).$
  \item[$\cdot$)] $\neg((q \lor \neg (p \rightarrow r)) \rightarrow (p \land (q \rightarrow r))).$
  \end{itemize}
  \begin{proof} Para este ejercicio analicemos los incisos por separado, esto es
    
    1. Encontrado la FND de la fórmula
    $\neg(w \rightarrow \neg p) \lor \neg ((\neg s \leftrightarrow w) \lor (p \land s))$,
    tenemos que si
    \[
    \varphi \equiv \neg(w \rightarrow \neg p) \lor \neg ((\neg s \leftrightarrow w) \lor (p \land s))
    \]
    luego
    \begin{eqnarray*}
      \varphi
      &\equiv_{1}& (w \land p) \lor \neg ((\neg s \leftrightarrow w) \lor (p \land s))\\
      &\equiv_{2}& (w \land p) \lor ( \neg (\neg s \leftrightarrow w) \land  \neg (p \land s))\\
      &\equiv_{3}& (w \land p) \lor ((s \leftrightarrow w) \land  (\neg p \lor \neg s))\\
      &\equiv_{4}& (w \land p) \lor (((s \land w) \lor (\neg s \land \neg w)) \land  (\neg p \lor \neg s))\\
      &\equiv_{5}& (w \land p) \lor ((\neg p \lor \neg s) \land (s \land w)) \lor ((\neg p \lor \neg s)
      \land (\neg s \land \neg w))\\
      &\equiv_{6}& (w \land p) \lor (\neg p \land (s \land w)) \lor (\neg s \land (s \land w))\\
      & & \hspace*{1.2cm} \lor (\neg p \land (\neg s \land \neg w)) \lor (\neg s \land (\neg s \land \neg w))\\
      &\equiv_{7}& (w \land p) \lor (\neg p \land s \land w) \lor ((\neg s \land s) \land w))\\
      & & \hspace*{1.2cm} \lor (\neg p \land \neg s \land \neg w) \lor (\neg s \land \neg s \land \neg w)\\
      &\equiv_{8}& (w \land p) \lor (\neg p \land s \land w) \lor (\bot \land w)\\
      & & \hspace*{1.2cm} \lor (\neg p \land \neg s \land \neg w) \lor ((\neg s \land \neg s) \land \neg w)\\
      &\equiv_{9}& (w \land p) \lor (\neg p \land s \land w) \lor \bot\\
      & & \hspace*{1.2cm} \lor (\neg p \land \neg s \land \neg w) \lor (\neg s \land \neg w)\\
      &\equiv_{10}& (w \land p) \lor (\neg p \land s \land w) \lor (\neg p \land \neg s \land \neg w)
      \lor (\neg s \land \neg w)\\
      &\equiv_{11}& (w \land p) \lor (\neg p \land s \land w)\lor (\neg s \land \neg w)
    \end{eqnarray*}
    1. Negación de la implicación.
    
    2. Leyes de De Morgan.
    
    3. Negación de bicondicional y leyes de De Morgan.
    
    4. Eliminación de la bicondicional.
    
    5. Distribución.
    
    6. Distribución.
    
    7. Asociación.
    
    8. Contradicción y asociación.
    
    9. Dominancia.
    
    10. Neutro e idempotencia.
    
    11. Absorción
    \[
    \therefore\;\; \neg(w \rightarrow \neg p) \lor \neg ((\neg s \leftrightarrow w) \lor (p \land s))
    \equiv (w \land p) \lor (\neg p \land s \land w)\lor (\neg s \land \neg w)
    \]
    \newpage
    2. Encontrando la FND de $\neg((q \lor \neg (p \rightarrow r)) \rightarrow (p \land (q \rightarrow r)))$,
    tenemos que si
    \[
    \phi \equiv \neg((q \lor \neg (p \rightarrow r)) \rightarrow (p \land (q \rightarrow r)))
    \]
    entonces:
    \begin{eqnarray*}
      \phi &\equiv_{1}& (q \lor \neg (p \rightarrow r)) \land \neg (p \land (q \rightarrow r))\\
      &\equiv_{2}& (q \lor (p \land \neg r)) \land (\neg p \lor \neg (q \rightarrow r))\\
      &\equiv_{3}& (q \lor (p \land \neg r)) \land (\neg p \lor (q \land \neg r))\\
      &\equiv_{4}& (q \land (\neg p \lor (q \land \neg r))) \lor ((p \land \neg r) \land (\neg p \lor (q \land \neg r)))\\
      &\equiv_{5}& (q \land \neg p) \lor (q \land \neg r) \lor (p \land \neg r \land \neg p)
      \lor (p \land \neg r \land q \land \neg r)\\
      &\equiv_{6}& (q \land \neg p) \lor (q \land \neg r) \lor (\bot \land \neg r) \lor (p \land \neg r \land q)\\
      &\equiv_{7}& (q \land \neg p) \lor (q \land \neg r) \lor \bot \lor (p \land \neg r \land q)\\
      &\equiv_{8}& (q \land \neg p) \lor (q \land \neg r) \lor (p \land \neg r \land q)\\
      &\equiv_{9}& (q \land \neg p) \lor (q \land \neg r)
    \end{eqnarray*}
    1. Negación de la implicación.
    
    2. Negación de la implicación y leyes de De Morgan.
    
    3. Negación de la implicación.
    
    4. Distributividad.
    
    5. Distributividad e idempotencia.
    
    6. Conmutatividad, idempotencia y contradicción.
    
    7. Dominancia.
    
    8. Neutro.
    
    9. Absorción.
    \[
    \therefore\;\; \phi \equiv (q \land \neg p) \lor (q \land \neg r)
    \]
  \end{proof}
\end{itemize}
\end{document}
